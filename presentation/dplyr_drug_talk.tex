% ---------------------------------------------------------------------------- %
% file: dplyr_drug_talk.Rnw
% author: Peter DeWitt <peter.dewitt@ucdenver.edu>
%
% presentation on dplyr, and as a result also magrittr, for the Denver R User
% Group (DRUG) MeetUp on 1 July 2014.
%
% ---------------------------------------------------------------------------- %

\documentclass{beamer}\usepackage[]{graphicx}\usepackage[]{color}
%% maxwidth is the original width if it is less than linewidth
%% otherwise use linewidth (to make sure the graphics do not exceed the margin)
\makeatletter
\def\maxwidth{ %
  \ifdim\Gin@nat@width>\linewidth
    \linewidth
  \else
    \Gin@nat@width
  \fi
}
\makeatother

\definecolor{fgcolor}{rgb}{0.345, 0.345, 0.345}
\newcommand{\hlnum}[1]{\textcolor[rgb]{0.686,0.059,0.569}{#1}}%
\newcommand{\hlstr}[1]{\textcolor[rgb]{0.192,0.494,0.8}{#1}}%
\newcommand{\hlcom}[1]{\textcolor[rgb]{0.678,0.584,0.686}{\textit{#1}}}%
\newcommand{\hlopt}[1]{\textcolor[rgb]{0,0,0}{#1}}%
\newcommand{\hlstd}[1]{\textcolor[rgb]{0.345,0.345,0.345}{#1}}%
\newcommand{\hlkwa}[1]{\textcolor[rgb]{0.161,0.373,0.58}{\textbf{#1}}}%
\newcommand{\hlkwb}[1]{\textcolor[rgb]{0.69,0.353,0.396}{#1}}%
\newcommand{\hlkwc}[1]{\textcolor[rgb]{0.333,0.667,0.333}{#1}}%
\newcommand{\hlkwd}[1]{\textcolor[rgb]{0.737,0.353,0.396}{\textbf{#1}}}%

\usepackage{framed}
\makeatletter
\newenvironment{kframe}{%
 \def\at@end@of@kframe{}%
 \ifinner\ifhmode%
  \def\at@end@of@kframe{\end{minipage}}%
  \begin{minipage}{\columnwidth}%
 \fi\fi%
 \def\FrameCommand##1{\hskip\@totalleftmargin \hskip-\fboxsep
 \colorbox{shadecolor}{##1}\hskip-\fboxsep
     % There is no \\@totalrightmargin, so:
     \hskip-\linewidth \hskip-\@totalleftmargin \hskip\columnwidth}%
 \MakeFramed {\advance\hsize-\width
   \@totalleftmargin\z@ \linewidth\hsize
   \@setminipage}}%
 {\par\unskip\endMakeFramed%
 \at@end@of@kframe}
\makeatother

\definecolor{shadecolor}{rgb}{.97, .97, .97}
\definecolor{messagecolor}{rgb}{0, 0, 0}
\definecolor{warningcolor}{rgb}{1, 0, 1}
\definecolor{errorcolor}{rgb}{1, 0, 0}
\newenvironment{knitrout}{}{} % an empty environment to be redefined in TeX

\usepackage{alltt}

% preamble%{{{
\usepackage{verbatim}

\author{Peter DeWitt\\peter.dewitt@ucdenver.edu}
\date{1 July 2014}
\title{Introduction to {\tt dplyr} and {\tt magrittr}}
\subtitle{Denver R Users Group\\www.meetup.com/DenverRUG}


%}}}
\IfFileExists{upquote.sty}{\usepackage{upquote}}{}
\begin{document}

% Title page, goals%{{{
\begin{frame}[fragile]
  \maketitle
\end{frame} 

\begin{frame}[fragile]
  \frametitle{Goals:}

  \begin{itemize}
    \item Showcase {\tt dplyr}, compare the ease of use compared to base R.
    \item Introduce the data manipulation grammar and philosophy behind {\tt
      dplyr}
    \item Illustrate the usefulness of the forward-piping operator which is
      part of {\tt dplyr} and extended further in {\tt magrittr}.  
  \end{itemize}

  \tableofcontents

\end{frame} 
%}}}

\section{{\tt dplyr}}%{{{
\begin{frame}[fragile]
  \frametitle{{\tt dplyr}: a grammar of data manipulation}
  \begin{itemize}
    \item Authored by Hadley Wickham and Romain Francois
    \item Current CRAN version 0.2

    \item<2-> Paraphrasing from a post on the RStudio blog
      \url{http://blog.rstudio.org/2014/01/17/introducing-dplyr}

      \begin{itemize}
        \item {\tt dplyr} is the next iteration of {\tt plyr}
        \item focuses only on {\tt data.frame}s
        \item faster, thanks in part to Francois work in {\tt Rcpp}, some use of
          multiple processors.
        \item improved API. 
        \item interface with remote database (PostgreSQL, MySQL, SQLite, and
          Google bigquery) tables using the same verbs for
          interacting with {\tt data.frame}s.  (Extendible to other backends)
        \item Common operations:
          \begin{itemize}
            \item {\tt group\_by}, {\tt summarize}, {\tt mutate}, {\tt filter},
              {\tt select}, and {\tt arrange}.
          \end{itemize}
      \end{itemize}

  \end{itemize}
\end{frame} 
%}}}

\subsection{Data Import}%{{{

\begin{frame}[fragile]
  \frametitle{Data Import}
  {\tt dplyr} does not have special tools for reading in data, but, if you need
  to {\tt rbind} sets together\ldots 



\begin{knitrout}\footnotesize
\definecolor{shadecolor}{rgb}{0.969, 0.969, 0.969}\color{fgcolor}\begin{kframe}
\begin{alltt}
\hlcom{# FAAs wildlife strikes on aircraft since 1990.  The data}
\hlcom{# can be downloaded, in a Microsoft Access DB,  from}
\hlcom{# http://www.faa.gov/airports/airport_safety/wildlife/database/}
\hlcom{# Tables in the DB were exported to csv files.  }
\hlcom{# A data dictionary, in an Excel file, was also}
\hlcom{# included in the download from faa.gov}

\hlcom{# column classes are set (in R code not shown) to ensure}
\hlcom{# that each column of the imported data is of the same class}
\hlstd{wls.90.99} \hlkwb{<-}
  \hlkwd{read.csv}\hlstd{(}\hlstr{"../data/STRIKE_REPORTS (1990-1999).csv"}\hlstd{,}
           \hlkwc{colClasses} \hlstd{= clclss)}
\hlstd{wls.00.09} \hlkwb{<-}
  \hlkwd{read.csv}\hlstd{(}\hlstr{"../data/STRIKE_REPORTS (2000-2009).csv"}\hlstd{,}
           \hlkwc{colClasses} \hlstd{= clclss)}
\hlstd{wls.10.14} \hlkwb{<-}
  \hlkwd{read.csv}\hlstd{(}\hlstr{"../data/STRIKE_REPORTS (2010-Current).csv"}\hlstd{,}
           \hlkwc{colClasses} \hlstd{= clclss)}
\end{alltt}
\end{kframe}
\end{knitrout}
\end{frame} 


\begin{frame}[fragile]
  \frametitle{Data Import}
\begin{knitrout}\footnotesize
\definecolor{shadecolor}{rgb}{0.969, 0.969, 0.969}\color{fgcolor}\begin{kframe}
\begin{alltt}
\hlcom{# Base does not require the columns to be of the same class,}
\hlcom{# only the same name}
\hlcom{# dplyr requires that the columnns are of the same class.}
\hlkwd{dim}\hlstd{(wls.90.99)}
\end{alltt}
\begin{verbatim}
## [1] 30150    94
\end{verbatim}
\begin{alltt}
\hlkwd{nrow}\hlstd{(wls.90.99)} \hlopt{+} \hlkwd{nrow}\hlstd{(wls.00.09)} \hlopt{+} \hlkwd{nrow}\hlstd{(wls.10.14)}
\end{alltt}
\begin{verbatim}
## [1] 142911
\end{verbatim}
\begin{alltt}
\hlstd{bnchmrk} \hlkwb{<-}
  \hlkwd{benchmark}\hlstd{(}\hlkwc{base} \hlstd{=} \hlkwd{rbind}\hlstd{(wls.90.99, wls.00.09, wls.10.14),}
            \hlkwc{dplyr} \hlstd{=} \hlkwd{rbind_list}\hlstd{(wls.90.99, wls.00.09, wls.10.14),}
            \hlkwc{replications} \hlstd{=} \hlnum{100}\hlstd{)}
\hlstd{bnchmrk[,} \hlkwd{c}\hlstd{(}\hlstr{"test"}\hlstd{,} \hlstr{"replications"}\hlstd{,} \hlstr{"elapsed"}\hlstd{,} \hlstr{"relative"}\hlstd{)]}
\end{alltt}
\begin{verbatim}
##    test replications elapsed relative
## 1  base          100   88.90    3.872
## 2 dplyr          100   22.96    1.000
\end{verbatim}
\end{kframe}
\end{knitrout}
\end{frame} 

\begin{frame}[fragile]
  \frametitle{Data Import}
\begin{knitrout}\footnotesize
\definecolor{shadecolor}{rgb}{0.969, 0.969, 0.969}\color{fgcolor}\begin{kframe}
\begin{alltt}
\hlstd{wls_df} \hlkwb{<-} \hlkwd{rbind}\hlstd{(wls.90.99, wls.00.09, wls.10.14)}
\hlkwd{class}\hlstd{(wls_df)}
\end{alltt}
\begin{verbatim}
## [1] "data.frame"
\end{verbatim}
\begin{alltt}
\hlstd{wls} \hlkwb{<-} \hlkwd{rbind_list}\hlstd{(wls.90.99, wls.00.09, wls.10.14)}
\hlkwd{class}\hlstd{(wls)}
\end{alltt}
\begin{verbatim}
## [1] "data.frame"
\end{verbatim}
\begin{alltt}
\hlcom{# A data frame tbl wraps a local data frame. The main}
\hlcom{# advantage to using a ‘tbl_df’ over a regular data frame is}
\hlcom{# the printing: tbl objects only print a few rows and all}
\hlcom{# the columns that fit on one screen, providing describing}
\hlcom{# the rest of it as text. [source: R help doc]}
\hlstd{wls_tbl_df} \hlkwb{<-} \hlkwd{tbl_df}\hlstd{(wls)}
\hlkwd{class}\hlstd{(wls_tbl_df)}
\end{alltt}
\begin{verbatim}
## [1] "tbl_df"     "tbl"        "data.frame"
\end{verbatim}
\end{kframe}
\end{knitrout}
\end{frame} 

\begin{frame}[fragile]
  \frametitle{Data Printing}
\begin{knitrout}\footnotesize
\definecolor{shadecolor}{rgb}{0.969, 0.969, 0.969}\color{fgcolor}\begin{kframe}
\begin{alltt}
\hlcom{# print(wls_df)  # takes a long time, not helpful}
\hlcom{# head(wls_df)   # two many columns to be useful}
\hlkwd{print}\hlstd{(wls_tbl_df,} \hlkwc{n} \hlstd{=} \hlnum{2}\hlstd{)}
\end{alltt}
\begin{verbatim}
## Source: local data frame [142,911 x 94]
## 
##    INDEX_NR OPID          OPERATOR     ATYPE AMA AMO EMA EMO AC_CLASS
## 1    100000  AAL AMERICAN AIRLINES     B-727 148  10  34  10        A
## 2    100001  UAL   UNITED AIRLINES B-737-300 148  24  10  01        A
## ..      ...  ...               ...       ... ... ... ... ...      ...
## Variables not shown: AC_MASS (int), NUM_ENGS (chr), TYPE_ENG (chr),
##   ENG_1_POS (chr), ENG_2_POS (int), ENG_3_POS (chr), ENG_4_POS (int), REG
##   (chr), FLT (chr), REMAINS_COLLECTED (lgl), REMAINS_SENT (lgl),
##   INCIDENT_DATE (chr), INCIDENT_MONTH (int), INCIDENT_YEAR (int),
##   TIME_OF_DAY (chr), TIME (int), AIRPORT_ID (chr), AIRPORT (chr), STATE
##   (chr), FAAREGION (chr), ENROUTE (chr), RUNWAY (chr), LOCATION (chr),
##   HEIGHT (int), SPEED (int), DISTANCE (dbl), PHASE_OF_FLT (chr), DAMAGE
##   (chr), STR_RAD (lgl), DAM_RAD (lgl), STR_WINDSHLD (lgl), DAM_WINDSHLD
##   (lgl), STR_NOSE (lgl), DAM_NOSE (lgl), STR_ENG1 (lgl), DAM_ENG1 (lgl),
##   STR_ENG2 (lgl), DAM_ENG2 (lgl), STR_ENG3 (lgl), DAM_ENG3 (lgl), STR_ENG4
##   (lgl), DAM_ENG4 (lgl), INGESTED (lgl), STR_PROP (lgl), DAM_PROP (lgl),
##   STR_WING_ROT (lgl), DAM_WING_ROT (lgl), STR_FUSE (lgl), DAM_FUSE (lgl),
##   STR_LG (lgl), DAM_LG (lgl), STR_TAIL (lgl), DAM_TAIL (lgl), STR_LGHTS
##   (lgl), DAM_LGHTS (lgl), STR_OTHER (lgl), DAM_OTHER (lgl), OTHER_SPECIFY
##   (chr), EFFECT (chr), EFFECT_OTHER (chr), SKY (chr), PRECIP (chr),
##   SPECIES_ID (chr), SPECIES (chr), BIRDS_SEEN (chr), BIRDS_STRUCK (chr),
##   SIZE (chr), WARNED (chr), COMMENTS (chr), REMARKS (chr), AOS (int),
##   COST_REPAIRS (int), COST_OTHER (int), COST_REPAIRS_INFL_ADJ (int),
##   COST_OTHER_INFL_ADJ (int), REPORTED_NAME (chr), REPORTED_TITLE (chr),
##   REPORTED_DATE (chr), SOURCE (chr), PERSON (chr), NR_INJURIES (int),
##   NR_FATALITIES (int), LUPDATE (chr), TRANSFER (lgl), INDICATED_DAMAGE
##   (lgl)
\end{verbatim}
\end{kframe}
\end{knitrout}
\end{frame} 
%}}}

\subsection{{\tt dplyr} verbs}%{{{
\begin{frame}[fragile]
  \frametitle{The verbs}
  \begin{itemize}
    \item ``Variable and function names should be lowercase. Use an underscore
      (\_) to separate words within a name. Generally, variable names should be
      nouns and function names should be verbs. Strive for names that are
      concise and meaningful (this is not easy!).'' - Hadley Wickham,
      \url{http://adv-r.had.co.nz/Style.html}

    \item Verbs in {\tt dplyr} 
      \begin{itemize}
        \item {\tt select},
        \item {\tt arrange},
        \item {\tt filter},
        \item {\tt mutate}, 
        \item {\tt summarize}.
      \end{itemize}
  \end{itemize}
\end{frame} 
%}}}

\subsubsection{select}%{{{

\begin{frame}[fragile]
  \frametitle{{\tt select}}
\begin{knitrout}\footnotesize
\definecolor{shadecolor}{rgb}{0.969, 0.969, 0.969}\color{fgcolor}\begin{kframe}
\begin{alltt}
\hlcom{# Select columns of a data.frame, tbl_df.}
\hlstd{wls_yr} \hlkwb{<-} \hlkwd{select}\hlstd{(wls_tbl_df, INCIDENT_YEAR, AIRPORT,}
                 \hlstd{ENG_1_POS, ENG_2_POS, DAM_ENG1, DAM_ENG2,}
                 \hlstd{HEIGHT, DISTANCE, SPEED)}
\hlkwd{print}\hlstd{(wls_yr,} \hlkwc{n} \hlstd{=} \hlnum{5}\hlstd{)}
\end{alltt}
\begin{verbatim}
## Source: local data frame [142,911 x 9]
## 
##    INCIDENT_YEAR                     AIRPORT ENG_1_POS ENG_2_POS DAM_ENG1
## 1           1992 DALLAS/FORT WORTH INTL ARPT         5         6    FALSE
## 2           1996             SACRAMENTO INTL         1         1    FALSE
## 3           1996         DENVER INTL AIRPORT         1         1    FALSE
## 4           1996             EPPLEY AIRFIELD         1         1    FALSE
## 5           1996 WASHINGTON DULLES INTL ARPT         1         1    FALSE
## ..           ...                         ...       ...       ...      ...
## Variables not shown: DAM_ENG2 (lgl), HEIGHT (int), DISTANCE (dbl), SPEED
##   (int)
\end{verbatim}
\end{kframe}
\end{knitrout}
\end{frame} 

\begin{frame}[fragile]
  \frametitle{{\tt select}}
\begin{knitrout}\footnotesize
\definecolor{shadecolor}{rgb}{0.969, 0.969, 0.969}\color{fgcolor}\begin{kframe}
\begin{alltt}
\hlcom{# relative speed }
\hlstd{bnch} \hlkwb{<-}
  \hlkwd{benchmark}\hlstd{(}\hlkwc{base}  \hlstd{= wls_tbl_df[,} \hlkwd{c}\hlstd{(}\hlstr{"INCIDENT_YEAR"}\hlstd{,} \hlstr{"AIRPORT"}\hlstd{,}
                                   \hlstr{"ENG_1_POS"}\hlstd{,} \hlstr{"ENG_2_POS"}\hlstd{,}
                                   \hlstr{"DAM_ENG1"}\hlstd{,} \hlstr{"DAM_ENG2"}\hlstd{,}
                                   \hlstr{"HEIGHT"}\hlstd{,} \hlstr{"DISTANCE"}\hlstd{,} \hlstr{"SPEED"}\hlstd{)],}
            \hlkwc{dplyr} \hlstd{=} \hlkwd{select}\hlstd{(wls_tbl_df,}
                           \hlstd{INCIDENT_YEAR, AIRPORT,}
                           \hlstd{ENG_1_POS, ENG_2_POS,}
                           \hlstd{DAM_ENG1, DAM_ENG2,}
                           \hlstd{HEIGHT, DISTANCE, SPEED),}
            \hlkwc{replications} \hlstd{=} \hlnum{100}\hlstd{)}
\hlkwd{select}\hlstd{(bnch, test, replications, elapsed, relative)}
\end{alltt}
\begin{verbatim}
##    test replications elapsed relative
## 1  base          100   0.008    1.000
## 2 dplyr          100   0.033    4.125
\end{verbatim}
\end{kframe}
\end{knitrout}

Selection of columns might be slower, but, there are some
tools to help speed up the coding, and maintenance.
\end{frame} 


\begin{frame}[fragile]
  \frametitle{{\tt select}}
\begin{knitrout}\footnotesize
\definecolor{shadecolor}{rgb}{0.969, 0.969, 0.969}\color{fgcolor}\begin{kframe}
\begin{alltt}
\hlcom{# ‘num_range("x", 1:5, width = 2)’: selects all variables}
\hlcom{# (numerically) from x01 to x05.}
\hlkwd{select}\hlstd{(wls_tbl_df,} \hlkwd{num_range}\hlstd{(}\hlstr{"DAM_ENG"}\hlstd{,} \hlnum{1}\hlopt{:}\hlnum{4}\hlstd{))}
\end{alltt}
\begin{verbatim}
## Source: local data frame [142,911 x 4]
## 
##    DAM_ENG1 DAM_ENG2 DAM_ENG3 DAM_ENG4
## 1     FALSE    FALSE    FALSE    FALSE
## 2     FALSE    FALSE    FALSE    FALSE
## 3     FALSE    FALSE    FALSE    FALSE
## 4     FALSE    FALSE    FALSE    FALSE
## 5     FALSE    FALSE    FALSE    FALSE
## 6     FALSE    FALSE    FALSE    FALSE
## 7     FALSE    FALSE    FALSE    FALSE
## 8     FALSE    FALSE    FALSE    FALSE
## 9     FALSE    FALSE    FALSE    FALSE
## 10    FALSE    FALSE    FALSE    FALSE
## ..      ...      ...      ...      ...
\end{verbatim}
\end{kframe}
\end{knitrout}
\end{frame} 

\begin{frame}[fragile]
  \frametitle{{\tt select}}
\begin{knitrout}\footnotesize
\definecolor{shadecolor}{rgb}{0.969, 0.969, 0.969}\color{fgcolor}\begin{kframe}
\begin{alltt}
\hlcom{# ‘starts_with(x, ignore.case = FALSE)’: names starts with ‘x’}
\hlkwd{select}\hlstd{(wls_tbl_df,} \hlkwd{starts_with}\hlstd{(}\hlstr{"DAM"}\hlstd{))}
\end{alltt}
\begin{verbatim}
## Source: local data frame [142,911 x 15]
## 
##    DAMAGE DAM_RAD DAM_WINDSHLD DAM_NOSE DAM_ENG1 DAM_ENG2 DAM_ENG3
## 1       N   FALSE        FALSE    FALSE    FALSE    FALSE    FALSE
## 2           FALSE        FALSE    FALSE    FALSE    FALSE    FALSE
## 3           FALSE        FALSE    FALSE    FALSE    FALSE    FALSE
## 4       N   FALSE        FALSE    FALSE    FALSE    FALSE    FALSE
## 5       N   FALSE        FALSE    FALSE    FALSE    FALSE    FALSE
## 6       M   FALSE        FALSE    FALSE    FALSE    FALSE    FALSE
## 7       N   FALSE        FALSE    FALSE    FALSE    FALSE    FALSE
## 8      M?   FALSE        FALSE    FALSE    FALSE    FALSE    FALSE
## 9       N   FALSE        FALSE    FALSE    FALSE    FALSE    FALSE
## 10          FALSE        FALSE    FALSE    FALSE    FALSE    FALSE
## ..    ...     ...          ...      ...      ...      ...      ...
## Variables not shown: DAM_ENG4 (lgl), DAM_PROP (lgl), DAM_WING_ROT (lgl),
##   DAM_FUSE (lgl), DAM_LG (lgl), DAM_TAIL (lgl), DAM_LGHTS (lgl), DAM_OTHER
##   (lgl)
\end{verbatim}
\end{kframe}
\end{knitrout}
\end{frame} 

\begin{frame}[fragile]
  \frametitle{{\tt select}}
\begin{knitrout}\footnotesize
\definecolor{shadecolor}{rgb}{0.969, 0.969, 0.969}\color{fgcolor}\begin{kframe}
\begin{alltt}
\hlcom{# ‘ends_with(x, ignore.case = FALSE)’: names ends in ‘x’}
\hlkwd{select}\hlstd{(wls_tbl_df,} \hlkwd{ends_with}\hlstd{(}\hlstr{"4"}\hlstd{))}
\end{alltt}
\begin{verbatim}
## Source: local data frame [142,911 x 2]
## 
##    STR_ENG4 DAM_ENG4
## 1     FALSE    FALSE
## 2     FALSE    FALSE
## 3     FALSE    FALSE
## 4     FALSE    FALSE
## 5     FALSE    FALSE
## 6     FALSE    FALSE
## 7     FALSE    FALSE
## 8     FALSE    FALSE
## 9     FALSE    FALSE
## 10    FALSE    FALSE
## ..      ...      ...
\end{verbatim}
\end{kframe}
\end{knitrout}
\end{frame} 

\begin{frame}[fragile]
  \frametitle{{\tt select}}
\begin{knitrout}\footnotesize
\definecolor{shadecolor}{rgb}{0.969, 0.969, 0.969}\color{fgcolor}\begin{kframe}
\begin{alltt}
\hlcom{# ‘matches(x, ignore.case = FALSE)’: selects all variables}
\hlcom{# whose name matches the regular expression ‘x’}
\hlkwd{select}\hlstd{(wls_tbl_df,} \hlkwd{matches}\hlstd{(}\hlstr{"ENG|DAM"}\hlstd{))}
\end{alltt}
\begin{verbatim}
## Source: local data frame [142,911 x 26]
## 
##    NUM_ENGS TYPE_ENG ENG_1_POS ENG_2_POS ENG_3_POS ENG_4_POS DAMAGE
## 1         3        D         5         6         5        NA      N
## 2         2        D         1         1                  NA       
## 3         2        D         1         1                  NA       
## 4         2        D         1         1                  NA      N
## 5         2        D         1         1                  NA      N
## 6         2        D         1         1                  NA      M
## 7         3        D         5         6         5        NA      N
## 8         2        C         4         4                  NA     M?
## 9         2        D         1         1                  NA      N
## 10        2        D         5         5                  NA       
## ..      ...      ...       ...       ...       ...       ...    ...
## Variables not shown: DAM_RAD (lgl), DAM_WINDSHLD (lgl), DAM_NOSE (lgl),
##   STR_ENG1 (lgl), DAM_ENG1 (lgl), STR_ENG2 (lgl), DAM_ENG2 (lgl), STR_ENG3
##   (lgl), DAM_ENG3 (lgl), STR_ENG4 (lgl), DAM_ENG4 (lgl), DAM_PROP (lgl),
##   DAM_WING_ROT (lgl), DAM_FUSE (lgl), DAM_LG (lgl), DAM_TAIL (lgl),
##   DAM_LGHTS (lgl), DAM_OTHER (lgl), INDICATED_DAMAGE (lgl)
\end{verbatim}
\end{kframe}
\end{knitrout}
\end{frame} 

\begin{frame}[fragile]
  \frametitle{{\tt select}}
\begin{knitrout}\footnotesize
\definecolor{shadecolor}{rgb}{0.969, 0.969, 0.969}\color{fgcolor}\begin{kframe}
\begin{alltt}
\hlcom{# ‘contains(x, ignore.case = FALSE)’: selects all}
\hlcom{# variables whose name  contains ‘x’}
\hlkwd{select}\hlstd{(wls_tbl_df,} \hlkwd{contains}\hlstd{(}\hlstr{"ENG"}\hlstd{))}
\end{alltt}
\begin{verbatim}
## Source: local data frame [142,911 x 14]
## 
##    NUM_ENGS TYPE_ENG ENG_1_POS ENG_2_POS ENG_3_POS ENG_4_POS STR_ENG1
## 1         3        D         5         6         5        NA    FALSE
## 2         2        D         1         1                  NA    FALSE
## 3         2        D         1         1                  NA     TRUE
## 4         2        D         1         1                  NA    FALSE
## 5         2        D         1         1                  NA    FALSE
## 6         2        D         1         1                  NA     TRUE
## 7         3        D         5         6         5        NA    FALSE
## 8         2        C         4         4                  NA    FALSE
## 9         2        D         1         1                  NA    FALSE
## 10        2        D         5         5                  NA    FALSE
## ..      ...      ...       ...       ...       ...       ...      ...
## Variables not shown: DAM_ENG1 (lgl), STR_ENG2 (lgl), DAM_ENG2 (lgl),
##   STR_ENG3 (lgl), DAM_ENG3 (lgl), STR_ENG4 (lgl), DAM_ENG4 (lgl)
\end{verbatim}
\end{kframe}
\end{knitrout}
\end{frame} 

\begin{frame}[fragile]
  \frametitle{{\tt select}}
  What about dropping variables?
\begin{knitrout}\footnotesize
\definecolor{shadecolor}{rgb}{0.969, 0.969, 0.969}\color{fgcolor}\begin{kframe}
\begin{alltt}
\hlkwd{print}\hlstd{(wls_yr,} \hlkwc{n} \hlstd{=} \hlnum{2}\hlstd{)}
\end{alltt}
\begin{verbatim}
## Source: local data frame [142,911 x 9]
## 
##    INCIDENT_YEAR                     AIRPORT ENG_1_POS ENG_2_POS DAM_ENG1
## 1           1992 DALLAS/FORT WORTH INTL ARPT         5         6    FALSE
## 2           1996             SACRAMENTO INTL         1         1    FALSE
## ..           ...                         ...       ...       ...      ...
## Variables not shown: DAM_ENG2 (lgl), HEIGHT (int), DISTANCE (dbl), SPEED
##   (int)
\end{verbatim}
\begin{alltt}
\hlkwd{print}\hlstd{(}\hlkwd{select}\hlstd{(wls_yr,} \hlopt{-}\hlstd{AIRPORT,} \hlopt{-}\hlkwd{starts_with}\hlstd{(}\hlstr{"ENG"}\hlstd{)),} \hlkwc{n} \hlstd{=} \hlnum{3}\hlstd{)}
\end{alltt}
\begin{verbatim}
## Source: local data frame [142,911 x 6]
## 
##    INCIDENT_YEAR DAM_ENG1 DAM_ENG2 HEIGHT DISTANCE SPEED
## 1           1992    FALSE    FALSE    300       NA   142
## 2           1996    FALSE    FALSE      0        0    NA
## 3           1996    FALSE    FALSE      0        0    NA
## ..           ...      ...      ...    ...      ...   ...
\end{verbatim}
\end{kframe}
\end{knitrout}
\end{frame} 
%}}}

\subsubsection{arrange}%{{{

\begin{frame}[fragile]
  \frametitle{{\tt arrange}}
arrange: reorder the rows. Multiple inputs are ordered from left-to-right.
\begin{knitrout}\footnotesize
\definecolor{shadecolor}{rgb}{0.969, 0.969, 0.969}\color{fgcolor}\begin{kframe}
\begin{alltt}
\hlstd{dat} \hlkwb{<-} \hlkwd{data.frame}\hlstd{(}\hlkwc{var1} \hlstd{=} \hlkwd{c}\hlstd{(}\hlnum{3}\hlstd{,} \hlnum{8}\hlstd{,} \hlnum{2}\hlstd{,} \hlnum{1}\hlstd{),}
                  \hlkwc{var2} \hlstd{=} \hlkwd{c}\hlstd{(}\hlstr{"E"}\hlstd{,} \hlstr{"A"}\hlstd{,} \hlstr{"A"}\hlstd{,} \hlstr{"B"}\hlstd{))}
\hlstd{dat}
\end{alltt}
\begin{verbatim}
##   var1 var2
## 1    3    E
## 2    8    A
## 3    2    A
## 4    1    B
\end{verbatim}
\begin{alltt}
\hlcom{# this would be very helpful for collecting data by a}
\hlcom{# subject id, visit number, ...}
\end{alltt}
\end{kframe}
\end{knitrout}
\end{frame} 

\begin{frame}[fragile]
  \frametitle{{\tt arrange}}
\begin{knitrout}\footnotesize
\definecolor{shadecolor}{rgb}{0.969, 0.969, 0.969}\color{fgcolor}\begin{kframe}
\begin{alltt}
\hlkwd{arrange}\hlstd{(dat, var2)}
\end{alltt}
\begin{verbatim}
##   var1 var2
## 1    8    A
## 2    2    A
## 3    1    B
## 4    3    E
\end{verbatim}
\begin{alltt}
\hlkwd{arrange}\hlstd{(dat, var2, var1)}
\end{alltt}
\begin{verbatim}
##   var1 var2
## 1    2    A
## 2    8    A
## 3    1    B
## 4    3    E
\end{verbatim}
\begin{alltt}
\hlcom{# this would be very helpful for collecting data by a}
\hlcom{# subject id, visit number, ...}
\end{alltt}
\end{kframe}
\end{knitrout}
\end{frame} 
%}}}

\subsubsection{filter}%{{{
\begin{frame}[fragile]
  \frametitle{{\tt filter}}
filter: return only a subset of the rows. If multiple
conditions are supplied they are combined with ‘\&’.
\begin{knitrout}\footnotesize
\definecolor{shadecolor}{rgb}{0.969, 0.969, 0.969}\color{fgcolor}\begin{kframe}
\begin{alltt}
\hlkwd{dim}\hlstd{(wls_yr)}
\end{alltt}
\begin{verbatim}
## [1] 142911      9
\end{verbatim}
\begin{alltt}
\hlkwd{filter}\hlstd{(wls_yr, INCIDENT_YEAR} \hlopt{>} \hlnum{2000}\hlstd{, INCIDENT_YEAR} \hlopt{<=} \hlnum{2005}\hlstd{)}
\end{alltt}
\begin{verbatim}
## Source: local data frame [31,947 x 9]
## 
##    INCIDENT_YEAR                       AIRPORT ENG_1_POS ENG_2_POS
## 1           2001           JOHN F KENNEDY INTL                  NA
## 2           2001       SAN FRANCISCO INTL ARPT         1         1
## 3           2001                  ORLANDO INTL         1         1
## 4           2001                  MOLOKAI ARPT         4         4
## 5           2001         LAMBERT-ST LOUIS INTL         5         5
## 6           2001              KANSAS CITY INTL         1         1
## 7           2001                       UNKNOWN         1         1
## 8           2001         AKRON-CANTON REGIONAL         7        NA
## 9           2001 DESTIN-FORT WALTON BEACH ARPT         5         5
## 10          2001           JOHN F KENNEDY INTL         1         1
## ..           ...                           ...       ...       ...
## Variables not shown: DAM_ENG1 (lgl), DAM_ENG2 (lgl), HEIGHT (int),
##   DISTANCE (dbl), SPEED (int)
\end{verbatim}
\end{kframe}
\end{knitrout}
\end{frame} 

\begin{frame}[fragile]
  \frametitle{{\tt filter}}
\begin{knitrout}\footnotesize
\definecolor{shadecolor}{rgb}{0.969, 0.969, 0.969}\color{fgcolor}\begin{kframe}
\begin{alltt}
\hlstd{bnch} \hlkwb{<-}
  \hlkwd{benchmark}\hlstd{(}\hlkwc{base} \hlstd{=} \hlkwd{subset}\hlstd{(wls_yr, INCIDENT_YEAR} \hlopt{>} \hlnum{2000} \hlopt{&} \hlstd{INCIDENT_YEAR} \hlopt{<=} \hlnum{2005}\hlstd{),}
            \hlkwc{dplyr} \hlstd{=} \hlkwd{filter}\hlstd{(wls_yr, INCIDENT_YEAR} \hlopt{>} \hlnum{2000}\hlstd{, INCIDENT_YEAR} \hlopt{<=} \hlnum{2005}\hlstd{),}
            \hlkwc{replications} \hlstd{=} \hlnum{100}\hlstd{)}
\hlkwd{select}\hlstd{(bnch, test, replications, elapsed, relative)}
\end{alltt}
\begin{verbatim}
##    test replications elapsed relative
## 1  base          100   8.015    5.475
## 2 dplyr          100   1.464    1.000
\end{verbatim}
\end{kframe}
\end{knitrout}
\end{frame} 
%}}}

\subsubsection{mutate}%{{{
\begin{frame}[fragile]
  \frametitle{{\tt mutate}}
mutate: add new columns. Multiple inputs create multiple columns.
\begin{knitrout}\footnotesize
\definecolor{shadecolor}{rgb}{0.969, 0.969, 0.969}\color{fgcolor}\begin{kframe}
\begin{alltt}
\hlstd{eng.lbls} \hlkwb{<-} \hlkwd{c}\hlstd{(}\hlstr{"mounted below the wing"}\hlstd{,} \hlstr{"mounted above the wing"}\hlstd{,}
              \hlstr{"part of the wing root"}\hlstd{,} \hlstr{"nacelle-mounted on the wing"}\hlstd{,}
              \hlstr{"mounted on the aft fuselage"}\hlstd{)}
\hlkwd{str}\hlstd{(}\hlkwd{mutate}\hlstd{(wls_yr,}
       \hlkwc{SPEED_MPH} \hlstd{= SPEED} \hlopt{*} \hlnum{1.15078}\hlstd{,} \hlcom{# SPEED was in knots}
       \hlkwc{ENG_1_POS} \hlstd{=} \hlkwd{factor}\hlstd{(ENG_1_POS,} \hlnum{19}\hlopt{:}\hlnum{23}\hlstd{, eng.lbls),}
       \hlkwc{ENG_2_POS} \hlstd{=} \hlkwd{factor}\hlstd{(ENG_2_POS,} \hlnum{19}\hlopt{:}\hlnum{23}\hlstd{, eng.lbls)))}
\end{alltt}
\begin{verbatim}
## Classes 'tbl_df', 'tbl' and 'data.frame':	142911 obs. of  10 variables:
##  $ INCIDENT_YEAR: int  1992 1996 1996 1996 1996 1996 1991 1993 1995 1990 ...
##  $ AIRPORT      : chr  "DALLAS/FORT WORTH INTL ARPT" "SACRAMENTO INTL" "DENVER INTL AIRPORT" "EPPLEY AIRFIELD" ...
##  $ ENG_1_POS    : Factor w/ 5 levels "mounted below the wing",..: NA NA NA NA NA NA NA NA NA NA ...
##  $ ENG_2_POS    : Factor w/ 5 levels "mounted below the wing",..: NA NA NA NA NA NA NA NA NA NA ...
##  $ DAM_ENG1     : logi  FALSE FALSE FALSE FALSE FALSE FALSE ...
##  $ DAM_ENG2     : logi  FALSE FALSE FALSE FALSE FALSE FALSE ...
##  $ HEIGHT       : int  300 0 0 0 1000 5000 0 1500 0 100 ...
##  $ DISTANCE     : num  NA 0 0 0 NA NA 0 NA 0 NA ...
##  $ SPEED        : int  142 NA NA NA NA NA 100 220 NA 135 ...
##  $ SPEED_MPH    : num  163 NA NA NA NA ...
\end{verbatim}
\end{kframe}
\end{knitrout}
\end{frame} 

\begin{frame}[fragile]
  \frametitle{{\tt mutate}}
\begin{knitrout}\footnotesize
\definecolor{shadecolor}{rgb}{0.969, 0.969, 0.969}\color{fgcolor}\begin{kframe}
\begin{alltt}
\hlstd{bnch} \hlkwb{<-}
\hlkwd{benchmark}\hlstd{(}\hlkwc{base} \hlstd{=} \hlkwd{within}\hlstd{(wls_yr, \{}
                        \hlstd{SPEED_MPH} \hlkwb{=} \hlstd{SPEED} \hlopt{*} \hlnum{1.15078}
                        \hlstd{ENG_1_POS} \hlkwb{=} \hlkwd{factor}\hlstd{(ENG_1_POS,} \hlnum{19}\hlopt{:}\hlnum{23}\hlstd{, eng.lbls)}
                        \hlstd{ENG_2_POS} \hlkwb{=} \hlkwd{factor}\hlstd{(ENG_2_POS,} \hlnum{19}\hlopt{:}\hlnum{23}\hlstd{, eng.lbls)\}),}
          \hlkwc{dplyr} \hlstd{=} \hlkwd{mutate}\hlstd{(wls_yr,}
                         \hlkwc{SPEED_MPH} \hlstd{= SPEED} \hlopt{*} \hlnum{1.15078}\hlstd{,}
                         \hlkwc{ENG_1_POS} \hlstd{=} \hlkwd{factor}\hlstd{(ENG_1_POS,} \hlnum{19}\hlopt{:}\hlnum{23}\hlstd{, eng.lbls),}
                         \hlkwc{ENG_2_POS} \hlstd{=} \hlkwd{factor}\hlstd{(ENG_2_POS,} \hlnum{19}\hlopt{:}\hlnum{23}\hlstd{, eng.lbls)),}
          \hlkwc{replications} \hlstd{=} \hlnum{100}\hlstd{)}
\hlkwd{select}\hlstd{(bnch, test, replications, elapsed, relative)}
\end{alltt}
\begin{verbatim}
##    test replications elapsed relative
## 1  base          100   5.445    1.007
## 2 dplyr          100   5.406    1.000
\end{verbatim}
\end{kframe}
\end{knitrout}
\end{frame} 
%}}}

\subsubsection{summarize}%{{{
\begin{frame}[fragile]
  \frametitle{{\tt summarize}}
summarise: reduce each group to a single row. Multiple
inputs create multiple output summaries.  (Two spellings:
{\tt summarize} and {\tt summarise}.)
\begin{knitrout}\footnotesize
\definecolor{shadecolor}{rgb}{0.969, 0.969, 0.969}\color{fgcolor}\begin{kframe}
\begin{alltt}
\hlkwd{summarise}\hlstd{(wls_yr,}
          \hlstr{"Mean speed"} \hlstd{=} \hlkwd{mean}\hlstd{(SPEED,} \hlkwc{na.rm} \hlstd{=} \hlnum{TRUE}\hlstd{),}
          \hlstr{"SD speed"}   \hlstd{=} \hlkwd{sd}\hlstd{(SPEED,} \hlkwc{na.rm} \hlstd{=} \hlnum{TRUE}\hlstd{),}
          \hlkwc{n}            \hlstd{=} \hlkwd{sum}\hlstd{(}\hlopt{!}\hlkwd{is.na}\hlstd{(SPEED)))}
\end{alltt}
\begin{verbatim}
## Source: local data frame [1 x 3]
## 
##   Mean speed SD speed     n
## 1      141.3    46.09 58938
\end{verbatim}
\end{kframe}
\end{knitrout}
\end{frame} 
%}}}

\subsubsection{group\_by}%{{{
\begin{frame}[fragile]
  \frametitle{{\tt group\_by}}
\begin{knitrout}\footnotesize
\definecolor{shadecolor}{rgb}{0.969, 0.969, 0.969}\color{fgcolor}\begin{kframe}
\begin{alltt}
\hlkwd{summarise}\hlstd{(}\hlkwd{group_by}\hlstd{(wls_yr, ENG_1_POS),}
          \hlstr{"Mean speed"} \hlstd{=} \hlkwd{mean}\hlstd{(SPEED,} \hlkwc{na.rm} \hlstd{=} \hlnum{TRUE}\hlstd{),}
          \hlstr{"SD speed"}   \hlstd{=} \hlkwd{sd}\hlstd{(SPEED,} \hlkwc{na.rm} \hlstd{=} \hlnum{TRUE}\hlstd{),}
          \hlkwc{n}            \hlstd{=} \hlkwd{sum}\hlstd{(}\hlopt{!}\hlkwd{is.na}\hlstd{(SPEED)))}
\end{alltt}
\begin{verbatim}
## Source: local data frame [11 x 4]
## 
##    ENG_1_POS Mean speed SD speed     n
## 1                113.20    40.40  1303
## 2          1     154.81    43.14 27634
## 3          2      62.00    43.39     4
## 4          3     108.32    33.48    31
## 5          4     123.90    41.74  7953
## 6          5     143.80    42.31 17701
## 7          6      99.06    36.67   482
## 8          7      83.91    29.34  3829
## 9          A      90.00       NA     1
## 10         C        NaN       NA     0
## 11         T        NaN       NA     0
\end{verbatim}
\end{kframe}
\end{knitrout}
\end{frame} 

\begin{frame}[fragile]
  \frametitle{{\tt group\_by}}
\begin{knitrout}\footnotesize
\definecolor{shadecolor}{rgb}{0.969, 0.969, 0.969}\color{fgcolor}\begin{kframe}
\begin{alltt}
\hlstd{bnch} \hlkwb{<-}
  \hlkwd{benchmark}\hlstd{(}\hlkwc{base} \hlstd{=} \hlkwd{aggregate}\hlstd{(SPEED} \hlopt{~} \hlstd{ENG_1_POS, wls_yr,}
                             \hlkwa{function}\hlstd{(}\hlkwc{x}\hlstd{)} \hlkwd{c}\hlstd{(}\hlkwc{mean} \hlstd{=} \hlkwd{mean}\hlstd{(x,} \hlkwc{na.rm} \hlstd{=} \hlnum{TRUE}\hlstd{),}
                                           \hlkwc{sd}   \hlstd{=} \hlkwd{sd}\hlstd{(x,} \hlkwc{na.rm} \hlstd{= T),}
                                           \hlkwc{n}    \hlstd{=} \hlkwd{sum}\hlstd{(}\hlopt{!}\hlkwd{is.na}\hlstd{(x)))),}
            \hlkwc{dplyr} \hlstd{=} \hlkwd{summarise}\hlstd{(}\hlkwd{group_by}\hlstd{(wls_yr, ENG_1_POS),}
                              \hlstr{"Mean speed"} \hlstd{=} \hlkwd{mean}\hlstd{(SPEED,} \hlkwc{na.rm} \hlstd{=} \hlnum{TRUE}\hlstd{),}
                              \hlstr{"SD speed"}   \hlstd{=} \hlkwd{sd}\hlstd{(SPEED,} \hlkwc{na.rm} \hlstd{=} \hlnum{TRUE}\hlstd{),}
                              \hlkwc{n}            \hlstd{=} \hlkwd{sum}\hlstd{(}\hlopt{!}\hlkwd{is.na}\hlstd{(SPEED))),}
            \hlkwc{replications} \hlstd{=} \hlnum{100}\hlstd{)}
\hlkwd{select}\hlstd{(bnch, test, replications, elapsed, relative)}
\end{alltt}
\begin{verbatim}
##    test replications elapsed relative
## 1  base          100  79.864    53.56
## 2 dplyr          100   1.491     1.00
\end{verbatim}
\end{kframe}
\end{knitrout}
\end{frame} 
%}}}

\subsection{Chaining Work together}%{{{
\begin{frame}[fragile]
  \frametitle{Say we need to filter, group\_by, and
  summarise data}
\begin{knitrout}\footnotesize
\definecolor{shadecolor}{rgb}{0.969, 0.969, 0.969}\color{fgcolor}\begin{kframe}
\begin{alltt}
\hlcom{# What is the mean distance from the airport, in kilometers, where the}
\hlcom{# strike took place, by damage to engine, on twin engine aircraft, between}
\hlcom{# 2002 and 2010, inclusive?}
\hlkwd{summarize}\hlstd{(}\hlkwd{group_by}\hlstd{(}\hlkwd{mutate}\hlstd{(}\hlkwd{filter}\hlstd{(wls, INCIDENT_YEAR} \hlopt{>=} \hlnum{2002}\hlstd{, INCIDENT_YEAR} \hlopt{<=}
    \hlnum{2010}\hlstd{, NUM_ENGS} \hlopt{==} \hlnum{2}\hlstd{),} \hlkwc{DISTANCE_KM} \hlstd{= DISTANCE} \hlopt{*} \hlnum{1.60934}\hlstd{), DAM_ENG1, DAM_ENG2),}
    \hlkwc{`mean distance in KM`} \hlstd{=} \hlkwd{mean}\hlstd{(DISTANCE_KM,} \hlkwc{na.rm} \hlstd{=} \hlnum{TRUE}\hlstd{))}
\end{alltt}
\begin{verbatim}
## Source: local data frame [4 x 3]
## Groups: DAM_ENG1
## 
##   DAM_ENG1 DAM_ENG2 mean distance in KM
## 1    FALSE    FALSE              1.3715
## 2    FALSE     TRUE              1.3228
## 3     TRUE    FALSE              0.8347
## 4     TRUE     TRUE              0.6584
\end{verbatim}
\begin{alltt}
\hlcom{# Without a comment to explain, how long would it take to explain the above}
\hlcom{# code?  You need to read from the inside out.  THERE IS A BETTER WAY!}
\end{alltt}
\end{kframe}
\end{knitrout}
\end{frame} 

\begin{frame}[fragile]
  \frametitle{Chain together multiple operations.}
\begin{knitrout}\footnotesize
\definecolor{shadecolor}{rgb}{0.969, 0.969, 0.969}\color{fgcolor}\begin{kframe}
\begin{alltt}
\hlstd{wls} \hlopt
\hlkwd{filter}\hlstd{(INCIDENT_YEAR} \hlopt{>=} \hlnum{2002}\hlstd{,}
       \hlstd{INCIDENT_YEAR} \hlopt{<=} \hlnum{2010}\hlstd{,}
       \hlstd{NUM_ENGS}      \hlopt{==} \hlnum{2}\hlstd{)} \hlopt
\hlkwd{mutate}\hlstd{(}\hlkwc{DISTANCE_KM} \hlstd{= DISTANCE} \hlopt{*} \hlnum{1.60934}\hlstd{)} \hlopt
\hlkwd{group_by}\hlstd{(DAM_ENG1, DAM_ENG2)} \hlopt
\hlkwd{summarise}\hlstd{(}\hlstr{"mean distance in KM"} \hlstd{=} \hlkwd{mean}\hlstd{(DISTANCE_KM,} \hlkwc{na.rm} \hlstd{=} \hlnum{TRUE}\hlstd{))}
\end{alltt}
\begin{verbatim}
## Source: local data frame [4 x 3]
## Groups: DAM_ENG1
## 
##   DAM_ENG1 DAM_ENG2 mean distance in KM
## 1    FALSE    FALSE              1.3715
## 2    FALSE     TRUE              1.3228
## 3     TRUE    FALSE              0.8347
## 4     TRUE     TRUE              0.6584
\end{verbatim}
\end{kframe}
\end{knitrout}
More detailed examples of the forward-piping operator
follow.
\end{frame} 
%}}}

\subsection{Joins}%{{{
\begin{frame}[fragile]
  \frametitle{joining data sets}
  \begin{itemize}
    \item {\tt dplyr} version 0.2 has the following joins:
      \begin{itemize}
        \item {\tt inner\_join}, 
        \item {\tt left\_join}, 
        \item {\tt semi\_join}, and
        \item {\tt anti\_join}.
      \end{itemize}
    \item Stated milestone for version 0.3 includes
      \begin{itemize}
        \item {\tt outer\_join}, 
        \item {\tt right\_join}, and
        \item {\tt cross\_join}.
      \end{itemize}
  \end{itemize}
\end{frame} 

\begin{frame}[fragile]
  \frametitle{joining data sets} 
  \framesubtitle{Data sets for examples:}
\begin{knitrout}\footnotesize
\definecolor{shadecolor}{rgb}{0.969, 0.969, 0.969}\color{fgcolor}\begin{kframe}
\begin{alltt}
\hlcom{# Baseball data from Lahman}
\hlstd{batting_df}  \hlkwb{<-} \hlkwd{data}\hlstd{(}\hlstr{"Batting"}\hlstd{,}  \hlkwc{package} \hlstd{=} \hlstr{"Lahman"}\hlstd{)}
\hlstd{pitching_df} \hlkwb{<-} \hlkwd{data}\hlstd{(}\hlstr{"Pitching"}\hlstd{,} \hlkwc{package} \hlstd{=} \hlstr{"Lahman"}\hlstd{)}
\hlstd{person_df}   \hlkwb{<-} \hlkwd{data}\hlstd{(}\hlstr{"Master"}\hlstd{,}   \hlkwc{package} \hlstd{=} \hlstr{"Lahman"}\hlstd{)}
\hlstd{batting_df}  \hlkwb{<-} \hlstd{Batting}  \hlopt \hlkwd{tbl_df}\hlstd{()}
\hlstd{pitching_df} \hlkwb{<-} \hlstd{Pitching} \hlopt \hlkwd{tbl_df}\hlstd{()}
\hlstd{person_df}   \hlkwb{<-} \hlstd{Master}   \hlopt \hlkwd{tbl_df}\hlstd{()}
\hlkwd{print}\hlstd{(batting_df,}  \hlkwc{n} \hlstd{=} \hlnum{3}\hlstd{)}
\end{alltt}
\begin{verbatim}
## Source: local data frame [96,600 x 24]
## 
##     playerID yearID stint teamID lgID  G G_batting AB R H X2B X3B HR RBI
## 1  aardsda01   2004     1    SFN   NL 11        11  0 0 0   0   0  0   0
## 2  aardsda01   2006     1    CHN   NL 45        43  2 0 0   0   0  0   0
## 3  aardsda01   2007     1    CHA   AL 25         2  0 0 0   0   0  0   0
## ..       ...    ...   ...    ...  ... ..       ... .. . . ... ... .. ...
##    SB
## 1   0
## 2   0
## 3   0
## .. ..
## Variables not shown: CS (int), BB (int), SO (int), IBB (int), HBP (int),
##   SH (int), SF (int), GIDP (int), G_old (int)
\end{verbatim}
\begin{alltt}
\hlkwd{print}\hlstd{(pitching_df,} \hlkwc{n} \hlstd{=} \hlnum{3}\hlstd{)}
\end{alltt}
\begin{verbatim}
## Source: local data frame [41,857 x 30]
## 
##     playerID yearID stint teamID lgID W L  G GS CG SHO SV IPouts  H ER HR
## 1  aardsda01   2004     1    SFN   NL 1 0 11  0  0   0  0     32 20  8  1
## 2  aardsda01   2006     1    CHN   NL 3 0 45  0  0   0  0    159 41 24  9
## 3  aardsda01   2007     1    CHA   AL 2 1 25  0  0   0  0     97 39 23  4
## ..       ...    ...   ...    ...  ... . . .. .. .. ... ..    ... .. .. ..
## Variables not shown: BB (int), SO (int), BAOpp (dbl), ERA (dbl), IBB
##   (int), WP (int), HBP (int), BK (int), BFP (int), GF (int), R (int), SH
##   (lgl), SF (lgl), GIDP (lgl)
\end{verbatim}
\begin{alltt}
\hlkwd{print}\hlstd{(person_df,}   \hlkwc{n} \hlstd{=} \hlnum{3}\hlstd{)}
\end{alltt}
\begin{verbatim}
## Source: local data frame [18,125 x 35]
## 
##    lahmanID  playerID managerID      hofID birthYear birthMonth birthDay
## 1         1 aaronha01        NA aaronha01h      1934          2        5
## 2         2 aaronto01        NA         NA      1939          8        5
## 3         3  aasedo01        NA         NA      1954          9        8
## ..      ...       ...       ...        ...       ...        ...      ...
## Variables not shown: birthCountry (chr), birthState (chr), birthCity
##   (chr), deathYear (int), deathMonth (int), deathDay (int), deathCountry
##   (chr), deathState (chr), deathCity (chr), nameFirst (chr), nameLast
##   (chr), nameNote (chr), nameGiven (chr), nameNick (chr), weight (int),
##   height (int), bats (fctr), throws (fctr), debut (date), finalGame
##   (date), college (chr), lahman40ID (chr), lahman45ID (chr), retroID
##   (chr), holtzID (chr), bbrefID (chr), deathDate (date), birthDate (date)
\end{verbatim}
\end{kframe}
\end{knitrout}
\end{frame} 

\begin{frame}[fragile]
  \frametitle{joining data sets}
  \framesubtitle{{\tt inner\_join}}
  Return all rows from x where there are matching values in y,
  and all columns from x and y. If there are multiple matches between
  x and y, all combination of the matches are returned.

\begin{knitrout}\footnotesize
\definecolor{shadecolor}{rgb}{0.969, 0.969, 0.969}\color{fgcolor}\begin{kframe}
\begin{alltt}
\hlcom{# build a data.frame for the pitching stats of players born in Colorado}
\hlstd{person_df} \hlopt
\hlkwd{filter}\hlstd{(birthState} \hlopt{==} \hlstr{"CO"}\hlstd{)} \hlopt
\hlkwd{select}\hlstd{(playerID)} \hlopt
\hlkwd{summarise}\hlstd{(}\hlkwd{n_distinct}\hlstd{(playerID))}        \hlcom{# n_distinct is a fast length(unique())}
\end{alltt}
\begin{verbatim}
## Source: local data frame [1 x 1]
## 
##   n_distinct(playerID)
## 1                   83
\end{verbatim}
\begin{alltt}
\hlstd{base_inner} \hlkwb{<-} \hlkwd{merge}\hlstd{(}\hlkwd{subset}\hlstd{(person_df, birthState} \hlopt{==} \hlstr{"CO"}\hlstd{),}
                    \hlstd{pitching_df,}
                    \hlkwc{by} \hlstd{=} \hlstr{"playerID"}\hlstd{,}
                    \hlkwc{all} \hlstd{=} \hlnum{FALSE}\hlstd{)} \hlopt
              \hlkwd{tbl_df}\hlstd{()}

\hlstd{dplyr_inner} \hlkwb{<-} \hlstd{person_df} \hlopt
               \hlkwd{filter}\hlstd{(birthState} \hlopt{==} \hlstr{"CO"}\hlstd{)} \hlopt
               \hlkwd{inner_join}\hlstd{(}\hlkwc{x} \hlstd{= ., pitching_df,} \hlkwc{by} \hlstd{=} \hlstr{"playerID"}\hlstd{)}

\hlstd{dplyr_inner} \hlopt \hlkwd{summarise}\hlstd{(}\hlkwd{n_distinct}\hlstd{(playerID))}
\end{alltt}
\begin{verbatim}
## Source: local data frame [1 x 1]
## 
##   n_distinct(playerID)
## 1                   54
\end{verbatim}
\begin{alltt}
\hlkwd{all.equal}\hlstd{(base_inner, dplyr_inner)}
\end{alltt}
\begin{verbatim}
## [1] TRUE
\end{verbatim}
\end{kframe}
\end{knitrout}
\end{frame}


\begin{frame}[fragile]
  \frametitle{joining data sets}
  \framesubtitle{{\tt inner\_join}}
\begin{knitrout}\footnotesize
\definecolor{shadecolor}{rgb}{0.969, 0.969, 0.969}\color{fgcolor}\begin{kframe}
\begin{alltt}
\hlstd{bnch} \hlkwb{<-}
  \hlkwd{benchmark}\hlstd{(}\hlkwc{base} \hlstd{=} \hlkwd{merge}\hlstd{(}\hlkwd{subset}\hlstd{(person_df, birthState} \hlopt{==} \hlstr{"CO"}\hlstd{),}
                         \hlstd{pitching_df,}
                         \hlkwc{by} \hlstd{=} \hlstr{"playerID"}\hlstd{,}
                         \hlkwc{all} \hlstd{=} \hlnum{FALSE}\hlstd{),}
            \hlkwc{dplyr} \hlstd{= person_df} \hlopt \hlkwd{filter}\hlstd{(birthState} \hlopt{==} \hlstr{"CO"}\hlstd{)} \hlopt
                    \hlkwd{inner_join}\hlstd{(}\hlkwc{x} \hlstd{= ., pitching_df,} \hlkwc{by} \hlstd{=} \hlstr{"playerID"}\hlstd{),}
            \hlkwc{replications} \hlstd{=} \hlnum{100}\hlstd{)}
\hlstd{bnch} \hlopt \hlkwd{select}\hlstd{(test, replications, elapsed, relative)}
\end{alltt}
\begin{verbatim}
##    test replications elapsed relative
## 1  base          100   4.104        9
## 2 dplyr          100   0.456        1
\end{verbatim}
\end{kframe}
\end{knitrout}
\end{frame} 

\begin{frame}[fragile]
  \frametitle{joining data sets}
  \framesubtitle{{\tt left\_join}} 
  Return all rows from x, and all columns from x and y. If
  there are multiple matches between x and y, all combination of the
  matches are returned.

\begin{knitrout}\footnotesize
\definecolor{shadecolor}{rgb}{0.969, 0.969, 0.969}\color{fgcolor}\begin{kframe}
\begin{alltt}
\hlstd{base_left} \hlkwb{<-} \hlkwd{merge}\hlstd{(}\hlkwd{subset}\hlstd{(person_df, birthState} \hlopt{==} \hlstr{"CO"}\hlstd{),}
                   \hlstd{pitching_df,}
                   \hlkwc{by} \hlstd{=} \hlstr{"playerID"}\hlstd{,}
                   \hlkwc{all.x} \hlstd{=} \hlnum{TRUE}\hlstd{)} \hlopt
             \hlkwd{tbl_df}\hlstd{()}

\hlstd{dplyr_left} \hlkwb{<-} \hlstd{person_df} \hlopt
              \hlkwd{filter}\hlstd{(birthState} \hlopt{==} \hlstr{"CO"}\hlstd{)} \hlopt
              \hlkwd{left_join}\hlstd{(}\hlkwc{x} \hlstd{= ., pitching_df,} \hlkwc{by} \hlstd{=} \hlstr{"playerID"}\hlstd{)}

\hlstd{dplyr_left} \hlopt \hlkwd{summarise}\hlstd{(}\hlkwd{n_distinct}\hlstd{(playerID))}
\end{alltt}
\begin{verbatim}
## Source: local data frame [1 x 1]
## 
##   n_distinct(playerID)
## 1                   83
\end{verbatim}
\begin{alltt}
\hlkwd{all.equal}\hlstd{(base_left, dplyr_left)}
\end{alltt}
\begin{verbatim}
## [1] TRUE
\end{verbatim}
\end{kframe}
\end{knitrout}
\end{frame} 

\begin{frame}[fragile]
  \frametitle{joining data sets}
  \framesubtitle{{\tt left\_join}}
\begin{knitrout}\footnotesize
\definecolor{shadecolor}{rgb}{0.969, 0.969, 0.969}\color{fgcolor}\begin{kframe}
\begin{alltt}
\hlstd{bnch} \hlkwb{<-}
  \hlkwd{benchmark}\hlstd{(}\hlkwc{base} \hlstd{=} \hlkwd{merge}\hlstd{(}\hlkwd{subset}\hlstd{(person_df, birthState} \hlopt{==} \hlstr{"CO"}\hlstd{),}
                         \hlstd{pitching_df,}
                         \hlkwc{by} \hlstd{=} \hlstr{"playerID"}\hlstd{,}
                         \hlkwc{all.x} \hlstd{=} \hlnum{TRUE}\hlstd{),}
            \hlkwc{dplyr} \hlstd{= person_df} \hlopt \hlkwd{filter}\hlstd{(birthState} \hlopt{==} \hlstr{"CO"}\hlstd{)} \hlopt
                    \hlkwd{left_join}\hlstd{(}\hlkwc{x} \hlstd{= ., pitching_df,} \hlkwc{by} \hlstd{=} \hlstr{"playerID"}\hlstd{),}
            \hlkwc{replications} \hlstd{=} \hlnum{100}\hlstd{)}
\hlstd{bnch} \hlopt \hlkwd{select}\hlstd{(test, replications, elapsed, relative)}
\end{alltt}
\begin{verbatim}
##    test replications elapsed relative
## 1  base          100   4.411    3.227
## 2 dplyr          100   1.367    1.000
\end{verbatim}
\end{kframe}
\end{knitrout}
\end{frame} 

\begin{frame}[fragile]
  \frametitle{joining data sets}
  \framesubtitle{{\tt semi\_join}}
  Return all rows from x where there are matching values in y,
  keeping just columns from x.

  A semi join differs from an inner join because an inner join will
  return one row of ‘x’ for each matching row of ‘y’, where a semi
  join will never duplicate rows of ‘x’.

\begin{knitrout}\footnotesize
\definecolor{shadecolor}{rgb}{0.969, 0.969, 0.969}\color{fgcolor}\begin{kframe}
\begin{alltt}
\hlstd{dplyr_semi} \hlkwb{<-} \hlkwd{semi_join}\hlstd{(person_df} \hlopt \hlkwd{filter}\hlstd{(birthState} \hlopt{==} \hlstr{"CO"}\hlstd{),}
                        \hlstd{pitching_df,}
                        \hlkwc{by} \hlstd{=} \hlstr{"playerID"}\hlstd{)}
\hlstd{dplyr_inner} \hlopt \hlkwd{dim}\hlstd{()}
\end{alltt}
\begin{verbatim}
## [1] 297  64
\end{verbatim}
\begin{alltt}
\hlstd{dplyr_semi} \hlopt \hlkwd{dim}\hlstd{()}
\end{alltt}
\begin{verbatim}
## [1] 54 35
\end{verbatim}
\begin{alltt}
\hlcom{# the semi join returns a subset of the person_df data.frame which has at least}
\hlcom{# one match in the pitching_df.}
\end{alltt}
\end{kframe}
\end{knitrout}

\end{frame} 

\begin{frame}[fragile]
  \frametitle{joining data sets}
  \framesubtitle{{\tt anti\_join}}
  Return all rows from x where there are not matching values in
  y, keeping just columns from x

\begin{knitrout}\footnotesize
\definecolor{shadecolor}{rgb}{0.969, 0.969, 0.969}\color{fgcolor}\begin{kframe}
\begin{alltt}
\hlstd{dplyr_left} \hlopt \hlkwd{summarise}\hlstd{(}\hlkwd{n_distinct}\hlstd{(playerID))}
\end{alltt}
\begin{verbatim}
## Source: local data frame [1 x 1]
## 
##   n_distinct(playerID)
## 1                   83
\end{verbatim}
\begin{alltt}
\hlstd{dplyr_semi} \hlopt \hlkwd{summarise}\hlstd{(}\hlkwd{n_distinct}\hlstd{(playerID))}
\end{alltt}
\begin{verbatim}
## Source: local data frame [1 x 1]
## 
##   n_distinct(playerID)
## 1                   54
\end{verbatim}
\begin{alltt}
\hlcom{# there are 83 - 54 = 29 players who have no pitching stats... who are they?}

\hlstd{dplyr_anti} \hlkwb{<-} \hlkwd{anti_join}\hlstd{(person_df} \hlopt \hlkwd{filter}\hlstd{(birthState} \hlopt{==} \hlstr{"CO"}\hlstd{),}
                        \hlstd{pitching_df,}
                        \hlkwc{by} \hlstd{=} \hlstr{"playerID"}\hlstd{)}
\hlstd{dplyr_anti} \hlopt \hlkwd{dim}\hlstd{()}
\end{alltt}
\begin{verbatim}
## [1] 29 35
\end{verbatim}
\end{kframe}
\end{knitrout}
\end{frame} 

\begin{frame}[fragile]
  \frametitle{joining data sets}
  \framesubtitle{{\tt outer\_join}}
  Return all rows from x and y, keeping all columns of x and y.

  Not implimented in {\tt dplyr\_0.2}, will be implimented in {\tt dplyr\_0.3}.

\begin{knitrout}\footnotesize
\definecolor{shadecolor}{rgb}{0.969, 0.969, 0.969}\color{fgcolor}\begin{kframe}
\begin{alltt}
\hlstd{base_outer} \hlkwb{<-} \hlkwd{merge}\hlstd{(batting_df, pitching_df,}
                    \hlkwc{by} \hlstd{=} \hlstr{"playerID"}\hlstd{,} \hlkwc{all} \hlstd{=} \hlnum{TRUE}\hlstd{,}
                    \hlkwc{suffixes} \hlstd{=} \hlkwd{c}\hlstd{(}\hlstr{".batting"}\hlstd{,} \hlstr{".pitching"}\hlstd{))} \hlopt
              \hlkwd{tbl_df}\hlstd{()}

\hlcom{# workaround for dplyr_0.2, outer_join should be part of dplyr_0.3}
\hlstd{my_outer_join} \hlkwb{<-} \hlkwa{function}\hlstd{(}\hlkwc{dfx}\hlstd{,} \hlkwc{dfy}\hlstd{,} \hlkwc{suffixes} \hlstd{=} \hlkwd{c}\hlstd{(}\hlstr{".x"}\hlstd{,} \hlstr{".y"}\hlstd{),} \hlkwc{...}\hlstd{) \{}

  \hlcom{# d1 <- left_join(batting_df, pitching_df, by = "playerID")}
  \hlcom{# d2 <- left_join(pitching_df, batting_df, by = "playerID")}
  \hlstd{d1} \hlkwb{<-} \hlkwd{left_join}\hlstd{(dfx, dfy, ...)}
  \hlstd{d2} \hlkwb{<-} \hlkwd{left_join}\hlstd{(dfy, dfx, ...)}

  \hlkwd{names}\hlstd{(d1)} \hlkwb{<-} \hlkwd{names}\hlstd{(d1)} \hlopt
  \hlkwd{gsub}\hlstd{(}\hlstr{"\textbackslash{}\textbackslash{}.x"}\hlstd{, suffixes[}\hlnum{1}\hlstd{],} \hlkwc{x} \hlstd{= .)}  \hlopt
  \hlkwd{gsub}\hlstd{(}\hlstr{"\textbackslash{}\textbackslash{}.y"}\hlstd{, suffixes[}\hlnum{2}\hlstd{],} \hlkwc{x} \hlstd{= .)}
  \hlkwd{names}\hlstd{(d2)} \hlkwb{<-} \hlkwd{names}\hlstd{(d2)} \hlopt
  \hlkwd{gsub}\hlstd{(}\hlstr{"\textbackslash{}\textbackslash{}.y"}\hlstd{, suffixes[}\hlnum{1}\hlstd{],} \hlkwc{x} \hlstd{= .)}  \hlopt
  \hlkwd{gsub}\hlstd{(}\hlstr{"\textbackslash{}\textbackslash{}.x"}\hlstd{, suffixes[}\hlnum{2}\hlstd{],} \hlkwc{x} \hlstd{= .)}

  \hlstd{out} \hlkwb{<-} \hlstd{dplyr::}\hlkwd{union}\hlstd{(d1, d2)}
  \hlkwd{return}\hlstd{(out)}
\hlstd{\}}

\hlstd{dplyr_outer} \hlkwb{<-} \hlkwd{my_outer_join}\hlstd{(batting_df, pitching_df,}
                             \hlkwd{c}\hlstd{(}\hlstr{".batting"}\hlstd{,} \hlstr{".pitching"}\hlstd{),}
                             \hlkwc{by} \hlstd{=} \hlstr{"playerID"}\hlstd{)}

\hlkwd{all.equal}\hlstd{(base_outer, dplyr_outer)}
\end{alltt}
\begin{verbatim}
## [1] TRUE
\end{verbatim}
\begin{alltt}
\hlkwd{dim}\hlstd{(dplyr_outer)}
\end{alltt}
\begin{verbatim}
## [1] 450488     53
\end{verbatim}
\end{kframe}
\end{knitrout}
\end{frame} 

\begin{frame}[fragile]
  \frametitle{joining data sets}
  \framesubtitle{{\tt outer\_join}}
  The workaround is still faster than using {\tt base::merge}!
\begin{knitrout}\footnotesize
\definecolor{shadecolor}{rgb}{0.969, 0.969, 0.969}\color{fgcolor}\begin{kframe}
\begin{alltt}
\hlstd{bnch} \hlkwb{<-}
  \hlkwd{benchmark}\hlstd{(}\hlkwc{base} \hlstd{= \{}
              \hlstd{base_outer} \hlkwb{<-} \hlkwd{merge}\hlstd{(batting_df, pitching_df,}
                                  \hlkwc{by} \hlstd{=} \hlstr{"playerID"}\hlstd{,} \hlkwc{all} \hlstd{=} \hlnum{TRUE}\hlstd{,}
                                  \hlkwc{suffixes} \hlstd{=} \hlkwd{c}\hlstd{(}\hlstr{".batting"}\hlstd{,} \hlstr{".pitching"}\hlstd{))}
                    \hlstd{\},}
            \hlkwc{dplyr} \hlstd{= \{}
              \hlstd{dplyr_outer} \hlkwb{<-} \hlkwd{my_outer_join}\hlstd{(batting_df, pitching_df,}
                                           \hlkwd{c}\hlstd{(}\hlstr{".batting"}\hlstd{,} \hlstr{".pitching"}\hlstd{),}
                                           \hlkwc{by} \hlstd{=} \hlstr{"playerID"}\hlstd{)}
            \hlstd{\},}
            \hlkwc{replications} \hlstd{=} \hlnum{10}\hlstd{)}

\hlstd{bnch} \hlopt \hlkwd{select}\hlstd{(test, replications, elapsed, relative)}
\end{alltt}
\begin{verbatim}
##    test replications elapsed relative
## 1  base           10   132.0    3.686
## 2 dplyr           10    35.8    1.000
\end{verbatim}
\end{kframe}
\end{knitrout}
\end{frame} 

\begin{frame}[fragile]
  \frametitle{joining data sets}
  \framesubtitle{{\tt cross\_join} and {\tt right\_join}}
  \begin{itemize}
    \item {\tt right\_join(y, x)} $\equiv$ {\tt left\_join(x, y)}
    \item {\tt cross\_join(x, y)}: every row of {\tt y} is matched with every row
      of {\tt x}.
  \end{itemize}
\begin{knitrout}\footnotesize
\definecolor{shadecolor}{rgb}{0.969, 0.969, 0.969}\color{fgcolor}\begin{kframe}
\begin{alltt}
\hlstd{dfx} \hlkwb{<-} \hlkwd{data.frame}\hlstd{(}\hlkwc{id} \hlstd{=} \hlnum{1}\hlopt{:}\hlnum{15}\hlstd{,} \hlkwc{var1} \hlstd{=} \hlkwd{rnorm}\hlstd{(}\hlnum{15}\hlstd{),} \hlkwc{var2} \hlstd{=} \hlkwd{runif}\hlstd{(}\hlnum{15}\hlstd{))}
\hlstd{dfy} \hlkwb{<-} \hlkwd{data.frame}\hlstd{(}\hlkwc{id} \hlstd{=} \hlnum{1}\hlopt{:}\hlnum{8}\hlstd{,} \hlkwc{var1} \hlstd{= LETTERS[}\hlnum{1}\hlopt{:}\hlnum{8}\hlstd{],} \hlkwc{var2} \hlstd{= letters[}\hlnum{1}\hlopt{:}\hlnum{8}\hlstd{])}

\hlstd{base_cross} \hlkwb{<-} \hlkwd{merge}\hlstd{(dfx, dfy,} \hlkwc{by} \hlstd{=} \hlkwa{NULL}\hlstd{)} \hlopt \hlkwd{tbl_df}\hlstd{()}

\hlcom{# my_cross_join, a function for cross joins via dplyr}
\hlstd{my_cross_join} \hlkwb{<-} \hlkwa{function}\hlstd{(}\hlkwc{dfx}\hlstd{,} \hlkwc{dfy}\hlstd{) \{}
  \hlstd{nrx} \hlkwb{<-} \hlkwd{nrow}\hlstd{(dfx)}

  \hlstd{dfy2} \hlkwb{<-} \hlkwd{replicate}\hlstd{(nrx, dfy,} \hlkwc{simplify} \hlstd{=} \hlnum{FALSE}\hlstd{)} \hlopt
          \hlkwd{rbind_all}\hlstd{()} \hlopt
          \hlkwd{mutate}\hlstd{(}\hlkwc{special.id} \hlstd{=} \hlkwd{rep}\hlstd{(}\hlnum{1}\hlopt{:}\hlstd{nrx,} \hlkwc{each} \hlstd{=} \hlkwd{nrow}\hlstd{(dfy)))}
  \hlstd{dfx2} \hlkwb{<-} \hlstd{dfx} \hlopt \hlkwd{mutate}\hlstd{(}\hlkwc{special.id} \hlstd{=} \hlnum{1}\hlopt{:}\hlstd{nrx)}
  \hlstd{out} \hlkwb{<-} \hlkwd{inner_join}\hlstd{(dfx2, dfy2,} \hlkwc{by} \hlstd{=} \hlstr{"special.id"}\hlstd{)} \hlopt \hlkwd{select}\hlstd{(}\hlopt{-}\hlstd{special.id)}
  \hlkwd{return}\hlstd{(out)}
\hlstd{\}}

\hlstd{dplyr_cross} \hlkwb{<-} \hlkwd{my_cross_join}\hlstd{(dfx, dfy)}

\hlkwd{all.equal}\hlstd{(base_cross, dplyr_cross)}
\end{alltt}
\begin{verbatim}
## [1] TRUE
\end{verbatim}
\end{kframe}
\end{knitrout}
\end{frame} 

\begin{frame}[fragile]
  \frametitle{joining data sets}
  \framesubtitle{{\tt cross\_join}} 
\begin{knitrout}\footnotesize
\definecolor{shadecolor}{rgb}{0.969, 0.969, 0.969}\color{fgcolor}\begin{kframe}
\begin{alltt}
\hlstd{dfx} \hlkwb{<-} \hlkwd{data.frame}\hlstd{(}\hlkwc{id} \hlstd{=} \hlnum{1}\hlopt{:}\hlnum{60}\hlstd{,} \hlkwc{var1} \hlstd{=} \hlkwd{rnorm}\hlstd{(}\hlnum{60}\hlstd{),} \hlkwc{var2} \hlstd{=} \hlkwd{runif}\hlstd{(}\hlnum{60}\hlstd{))}
\hlstd{dfy} \hlkwb{<-} \hlkwd{data.frame}\hlstd{(}\hlkwc{id} \hlstd{=} \hlnum{1}\hlopt{:}\hlnum{13}\hlstd{,} \hlkwc{var1} \hlstd{= LETTERS[}\hlnum{1}\hlopt{:}\hlnum{13}\hlstd{],} \hlkwc{var2} \hlstd{= letters[}\hlnum{1}\hlopt{:}\hlnum{13}\hlstd{])}

\hlstd{bnch} \hlkwb{<-}
  \hlkwd{benchmark}\hlstd{(}\hlkwc{base} \hlstd{= \{}
              \hlstd{base_cross} \hlkwb{<-} \hlkwd{merge}\hlstd{(dfx, dfy,} \hlkwc{by} \hlstd{=} \hlkwa{NULL}\hlstd{)} \hlopt \hlkwd{tbl_df}\hlstd{()}
            \hlstd{\},}
            \hlkwc{dplyr} \hlstd{= \{}
              \hlstd{dplyr_cross} \hlkwb{<-} \hlkwd{my_cross_join}\hlstd{(dfx, dfy)}
            \hlstd{\},}
            \hlkwc{replications} \hlstd{=} \hlnum{1000}\hlstd{)}

\hlstd{bnch} \hlopt \hlkwd{select}\hlstd{(test, replications, elapsed, relative)}
\end{alltt}
\begin{verbatim}
##    test replications elapsed relative
## 1  base         1000   4.511    1.508
## 2 dplyr         1000   2.991    1.000
\end{verbatim}
\end{kframe}
\end{knitrout}
  {\tt dplyr} is fast, fast enough to overcome the additional scripting.
\end{frame} 
%}}}


\subsection{Memory Usage}%{{{
\begin{frame}[fragile]
  \frametitle{Memory usage} 
\begin{knitrout}\footnotesize
\definecolor{shadecolor}{rgb}{0.969, 0.969, 0.969}\color{fgcolor}\begin{kframe}
\begin{alltt}
\hlstd{this_df2} \hlkwb{<-} \hlstd{this_df} \hlkwb{<-} \hlkwd{data.frame}\hlstd{(}\hlkwc{var1} \hlstd{=} \hlnum{1}\hlopt{:}\hlnum{5}\hlstd{,} \hlkwc{var2} \hlstd{=} \hlkwd{rnorm}\hlstd{(}\hlnum{5}\hlstd{))}
\hlkwd{changes}\hlstd{(this_df, this_df2)}
\end{alltt}
\begin{verbatim}
## <identical>
\end{verbatim}
\begin{alltt}
\hlstd{this_df}\hlopt{$}\hlstd{var1} \hlkwb{<-} \hlkwd{rexp}\hlstd{(}\hlnum{5}\hlstd{,} \hlkwc{rate} \hlstd{=} \hlnum{2}\hlstd{)}
\hlkwd{changes}\hlstd{(this_df, this_df2)}
\end{alltt}
\begin{verbatim}
## Changed variables:
##           old        new      
## var1      0x4fd0ce58 0x34e7010
## 
## Changed attributes:
##           old       new      
## row.names 0x7348e38 0x7349150
\end{verbatim}
\begin{alltt}
\hlkwd{location}\hlstd{(this_df2)}
\end{alltt}
\begin{verbatim}
## <0x2f459c68>
## Variables:
##  * var1:      <0x34e7010 >
##  * var2:      <0x4fd23750>
## Attributes:
##  * names:     <0x2f459d10>
##  * row.names: <0xa94a308 >
##  * class:     <0x13f6af08>
\end{verbatim}
\end{kframe}
\end{knitrout}
  Prior to 3.0.1 the edit to {\tt var1} would have created a deep copy of {\tt
  this\_df}.
\end{frame} 

\begin{frame}[fragile]
  \frametitle{Memory usage}
\begin{knitrout}\footnotesize
\definecolor{shadecolor}{rgb}{0.969, 0.969, 0.969}\color{fgcolor}\begin{kframe}
\begin{alltt}
\hlstd{this_df2} \hlkwb{<-} \hlstd{this_df} \hlkwb{<-} \hlkwd{data.frame}\hlstd{(}\hlkwc{var1} \hlstd{=} \hlnum{1}\hlopt{:}\hlnum{5}\hlstd{,} \hlkwc{var2} \hlstd{=} \hlkwd{rnorm}\hlstd{(}\hlnum{5}\hlstd{))} \hlopt \hlkwd{tbl_df}\hlstd{()}
\hlkwd{changes}\hlstd{(this_df, this_df2)}
\end{alltt}
\begin{verbatim}
## <identical>
\end{verbatim}
\begin{alltt}
\hlstd{this_df} \hlkwb{<-} \hlstd{this_df} \hlopt \hlkwd{mutate}\hlstd{(}\hlkwc{var1} \hlstd{=} \hlkwd{rexp}\hlstd{(}\hlnum{5}\hlstd{))}
\hlkwd{changes}\hlstd{(this_df, this_df2)}
\end{alltt}
\begin{verbatim}
## Changed variables:
##           old        new       
## var1      0x159f99e8 0x3ef6a910
## 
## Changed attributes:
##           old        new      
## names     0x53bbf10  0x5981f08
## class     0x47036b68 0x620e5d0
## row.names 0xce43c08  0xce43f20
\end{verbatim}
\end{kframe}
\end{knitrout}
  {\tt dplyr} ``smart enough to create only one new column: all the other
  columns continue to point at their old locations.''
\end{frame} 

\begin{frame}[fragile]
  \frametitle{{\tt dplyr} memory usage}
  From the {\tt vignette("memory", "dplyr")}
  \begin{itemize}
    \item {\tt tbl\_df()} and {\tt group\_by()} don't copy columns
    \item {\tt select()} never copies columns, even when you rename them
    \item {\tt mutate()} never copies columns, except when you modify an existing column
    \item {\tt arrange()} must copy because you're changing
      the order of every column. This is an expensive
      operation for big data, but you can generally avoid it
      using the order argument to window functions
    \item {\tt summarise()} creates new data, but it's
      usually at least an order of magnitude smaller than
      the original data.
  \end{itemize}
\end{frame} 
%}}}

\subsection{Window Functions}%{{{
\begin{frame}[fragile]
  \frametitle{Window Functions}
  \begin{itemize}
    \item See {\tt vignette("window-functions", package = "dplyr")}

    \item window functions are variations of aggreation functions.

      \begin{itemize}
        \item Aggregation functions such as {\tt sum()} and {\tt median()} are
          maps between $\mathbb{R}^n \rightarrow \mathbb{R}^1$.
        \item Window function are maps between $\mathbb{R}^n \rightarrow \mathbb{R}^n$.
          Examples: {\tt cumsum()}, {\tt rank()}, {\tt lag()}
      \end{itemize} 
  \end{itemize} 
\end{frame} 

\begin{frame}[fragile]
  \frametitle{Window Functions}
  \framesubtitle{Examples}
\begin{knitrout}\footnotesize
\definecolor{shadecolor}{rgb}{0.969, 0.969, 0.969}\color{fgcolor}\begin{kframe}
\begin{alltt}
\hlstd{batting} \hlkwb{<-} \hlstd{Batting} \hlopt \hlkwd{tbl_df}\hlstd{()} \hlopt
           \hlkwd{select}\hlstd{(playerID, yearID, teamID, G, AB}\hlopt{:}\hlstd{H, HR)}

\hlcom{# For each player, find the two years with most hits}
\hlstd{batting} \hlopt
  \hlkwd{group_by}\hlstd{(playerID)} \hlopt
  \hlkwd{filter}\hlstd{(}\hlkwd{min_rank}\hlstd{(}\hlkwd{desc}\hlstd{(H))} \hlopt{<=} \hlnum{2} \hlopt{&} \hlstd{H} \hlopt{>} \hlnum{0}\hlstd{)}
\end{alltt}
\begin{verbatim}
## Source: local data frame [24,834 x 8]
## Groups: playerID
## 
##     playerID yearID teamID   G  AB   R   H HR
## 1  aaronha01   1959    ML1 154 629 116 223 39
## 2  aaronha01   1963    ML1 161 631 121 201 44
## 3  aaronto01   1962    ML1 141 334  54  77  8
## 4  aaronto01   1968    ATL  98 283  21  69  1
## 5   abadan01   2003    BOS   9  17   1   2  0
## 6   abadfe01   2012    HOU  37   7   0   1  0
## 7  abadijo01   1875    PH3  11  45   3  10  0
## 8  abadijo01   1875    BR2   1   4   1   1  0
## 9  abbated01   1904    BSN 154 579  76 148  3
## 10 abbated01   1905    BSN 153 610  70 170  3
## ..       ...    ...    ... ... ... ... ... ..
\end{verbatim}
\begin{alltt}
\hlcom{# Within each player, rank each year by the number of games played}
\hlstd{batting} \hlopt
  \hlkwd{group_by}\hlstd{(playerID)} \hlopt
  \hlkwd{mutate}\hlstd{(}\hlkwc{G_rank} \hlstd{=} \hlkwd{min_rank}\hlstd{(G))}
\end{alltt}
\begin{verbatim}
## Source: local data frame [96,600 x 9]
## Groups: playerID
## 
##     playerID yearID teamID   G  AB   R   H HR G_rank
## 1  aardsda01   2004    SFN  11   0   0   0  0      2
## 2  aardsda01   2006    CHN  45   2   0   0  0      4
## 3  aardsda01   2007    CHA  25   0   0   0  0      3
## 4  aardsda01   2008    BOS  47   1   0   0  0      5
## 5  aardsda01   2009    SEA  73   0   0   0  0      7
## 6  aardsda01   2010    SEA  53   0   0   0  0      6
## 7  aardsda01   2012    NYA   1  NA  NA  NA NA      1
## 8  aaronha01   1954    ML1 122 468  58 131 13      4
## 9  aaronha01   1955    ML1 153 602 105 189 27     13
## 10 aaronha01   1956    ML1 153 609 106 200 26     13
## ..       ...    ...    ... ... ... ... ... ..    ...
\end{verbatim}
\begin{alltt}
\hlcom{# For each player, the proportion of years with more home runs than the prior}
\hlcom{# year}
\hlstd{batting} \hlopt
  \hlkwd{group_by}\hlstd{(playerID)} \hlopt
  \hlkwd{mutate}\hlstd{(}\hlkwc{more_hrs} \hlstd{= HR} \hlopt{>} \hlkwd{lag}\hlstd{(HR))} \hlopt                 \hlcom{# Window function}
  \hlkwd{summarize}\hlstd{(}\hlkwc{more_hrs} \hlstd{=} \hlkwd{mean}\hlstd{(more_hrs,} \hlkwc{na.rm} \hlstd{=} \hlnum{TRUE}\hlstd{))}  \hlcom{# aggregate function}
\end{alltt}
\begin{verbatim}
## Source: local data frame [17,908 x 2]
## 
##     playerID more_hrs
## 1  aardsda01   0.0000
## 2  aaronha01   0.4545
## 3  aaronto01   0.3333
## 4   aasedo01   0.0000
## 5   abadan01   0.0000
## 6   abadfe01   0.0000
## 7  abadijo01   0.0000
## 8  abbated01   0.2222
## 9  abbeybe01   0.0000
## 10 abbeych01   0.7500
## ..       ...      ...
\end{verbatim}
\begin{alltt}
\hlcom{# For each player, compute avg change in games played per year}
\hlstd{batting} \hlopt
  \hlkwd{group_by}\hlstd{(playerID)} \hlopt
  \hlkwd{mutate}\hlstd{(}\hlkwc{G_change} \hlstd{= (G} \hlopt{-} \hlkwd{lag}\hlstd{(G))} \hlopt{/} \hlstd{(yearID} \hlopt{-} \hlkwd{lag}\hlstd{(yearID)))}
\end{alltt}
\begin{verbatim}
## Source: local data frame [96,600 x 9]
## Groups: playerID
## 
##     playerID yearID teamID   G  AB   R   H HR G_change
## 1  aardsda01   2004    SFN  11   0   0   0  0       NA
## 2  aardsda01   2006    CHN  45   2   0   0  0       17
## 3  aardsda01   2007    CHA  25   0   0   0  0      -20
## 4  aardsda01   2008    BOS  47   1   0   0  0       22
## 5  aardsda01   2009    SEA  73   0   0   0  0       26
## 6  aardsda01   2010    SEA  53   0   0   0  0      -20
## 7  aardsda01   2012    NYA   1  NA  NA  NA NA      -26
## 8  aaronha01   1954    ML1 122 468  58 131 13       NA
## 9  aaronha01   1955    ML1 153 602 105 189 27       31
## 10 aaronha01   1956    ML1 153 609 106 200 26        0
## ..       ...    ...    ... ... ... ... ... ..      ...
\end{verbatim}
\begin{alltt}
\hlcom{# For each player, find all when they played more games than average}
\hlstd{batting} \hlopt
  \hlkwd{group_by}\hlstd{(playerID)} \hlopt
  \hlkwd{filter}\hlstd{(G} \hlopt{>} \hlkwd{mean}\hlstd{(G))} \hlopt
  \hlkwd{select}\hlstd{(playerID, yearID)}
\end{alltt}
\begin{verbatim}
## Source: local data frame [47,769 x 2]
## Groups: playerID
## 
##     playerID yearID
## 1  aardsda01   2006
## 2  aardsda01   2008
## 3  aardsda01   2009
## 4  aardsda01   2010
## 5  aaronha01   1955
## 6  aaronha01   1956
## 7  aaronha01   1957
## 8  aaronha01   1958
## 9  aaronha01   1959
## 10 aaronha01   1960
## ..       ...    ...
\end{verbatim}
\end{kframe}
\end{knitrout}
\end{frame} 

%}}}

\subsection{Other data sources}%{{{
\begin{frame}[fragile]
  \frametitle{Other Data Sources}
  \begin{itemize}
    \item {\tt dplyr} works for 
      \begin{itemize}
        \item {\tt data.frame}s, \item {\tt data.table}s,
          databases, and multidimensional arrays.
        \item Same verbs used for all data sources.
        \item See {\tt vignette("databases", package =
          "dplyr")} for more details.

      \end{itemize}
  \end{itemize}
\end{frame} 

\begin{frame}[fragile]
  \frametitle{{\tt data.table} vs {\tt dplyr}}
  From the {\\ dplyr} introduction vignette:
  \begin{itemize}
    \item For multiple operations, data.table can be faster
      because you usually use it with multiple verbs at the
      same time. For example, with data table you can do a
      mutate and a select in a single step, and it's smart
      enough to know that there's no point in computing the
      new variable for the rows you're about to throw away.
    \item The advantages of using {\tt dplyr} with data
      tables are:
      \begin{itemize}
        \item For common data manipulation tasks, it
          insulates you from reference semantics of
          data.tables, and protects you from accidentally
          modifying your data.
        \item Instead of one complex method built on the
          subscripting operator ([), it provides many simple
          methods.
      \end{itemize} 
  \end{itemize}
\end{frame} 
%}}}

\section{magrittr}%{{{
\begin{frame}[fragile]
  \frametitle{{\tt magrittr}: a forward-pipe operator for {\tt R}}
  \framesubtitle{ceci n'est pas un pipe (this is not a pipe)}

  \begin{itemize}
    \item {\tt dplyr} functionality is made more powerful via the \verb|%>%|,
      or equivalently, \verb|\%.%$|, operator.

    \item Additional functionally provided by the {\tt magrittr} package
      authored by Stefan Bache and Hadley Wickham.

    \item These operators are similar to 
      \begin{itemize} 
        \item F\#'s $|>$, or
        \item Linux's $|$.
      \end{itemize}

    \item Use of these operators will drastically change your {\tt R} syntax.

    \item Helpful to writing complex, nested, operations.
    \item``Read from left to right instead of inside out.''

  \end{itemize}
\end{frame} 

\begin{frame}[fragile]
  \frametitle{{\tt magrittr}: a foward-pipe operator for {\tt R}}
  \framesubtitle{Examples} 

\begin{knitrout}\footnotesize
\definecolor{shadecolor}{rgb}{0.969, 0.969, 0.969}\color{fgcolor}\begin{kframe}
\begin{alltt}
\hlstd{mu} \hlkwb{<-} \hlnum{1}
\hlstd{sigma} \hlkwb{<-} \hlnum{4}
\hlstd{N} \hlkwb{<-} \hlnum{5}
\hlstd{y} \hlkwb{<-} \hlkwd{rnorm}\hlstd{(N, mu, sigma)}

\hlcom{# -2 log likelihood, standard nested operations, i.e, infix notation}
\hlopt{-}\hlnum{2} \hlopt{*} \hlkwd{log}\hlstd{((}\hlnum{1}\hlopt{/}\hlkwd{sqrt}\hlstd{(}\hlnum{2} \hlopt{*} \hlstd{pi} \hlopt{*} \hlstd{sigma}\hlopt{^}\hlnum{2}\hlstd{))}\hlopt{^}\hlstd{(N)} \hlopt{*} \hlkwd{exp}\hlstd{(}\hlopt{-}\hlnum{1}\hlopt{/}\hlstd{(}\hlnum{2} \hlopt{*} \hlstd{sigma}\hlopt{^}\hlnum{2}\hlstd{)} \hlopt{*} \hlkwd{sum}\hlstd{((y} \hlopt{-} \hlstd{mu)}\hlopt{^}\hlnum{2}\hlstd{)))}
\end{alltt}
\begin{verbatim}
## [1] 24.06
\end{verbatim}
\begin{alltt}
\hlcom{# -2 log likelihood, using forward-piping, somewhat like postfix notation}
\hlstd{y} \hlopt
\hlkwd{subtract}\hlstd{(mu)} \hlopt
\hlkwd{raise_to_power}\hlstd{(}\hlnum{2}\hlstd{)} \hlopt
\hlstd{sum} \hlopt
\hlkwd{divide_by}\hlstd{(}\hlopt{-}\hlnum{2} \hlopt{*} \hlstd{sigma}\hlopt{^}\hlnum{2}\hlstd{)} \hlopt
\hlstd{exp} \hlopt
\hlkwd{multiply_by}\hlstd{((}\hlnum{2} \hlopt{*} \hlstd{pi} \hlopt{*} \hlstd{sigma}\hlopt{^}\hlnum{2}\hlstd{)}\hlopt{^}\hlstd{(}\hlopt{-}\hlstd{N}\hlopt{/}\hlnum{2}\hlstd{))} \hlopt
\hlstd{log} \hlopt
\hlkwd{multiply_by}\hlstd{(}\hlopt{-}\hlnum{2}\hlstd{)}
\end{alltt}
\begin{verbatim}
## [1] 24.06
\end{verbatim}
\end{kframe}
\end{knitrout}
What do you think?  Pros and Cons?

\end{frame} 
%}}}

\section{Wrap Up}%{{{
\begin{frame}[fragile]
  \frametitle{Reproducibility}
  The data, code, sides, etc.\ all at \url{github.com/dewittpe/dplyr-demo}

\begin{knitrout}\footnotesize
\definecolor{shadecolor}{rgb}{0.969, 0.969, 0.969}\color{fgcolor}\begin{kframe}
\begin{alltt}
\hlkwd{print}\hlstd{(}\hlkwd{sessionInfo}\hlstd{(),} \hlkwc{locale} \hlstd{=} \hlnum{FALSE}\hlstd{)}
\end{alltt}
\begin{verbatim}
## R version 3.1.0 (2014-04-10)
## Platform: x86_64-pc-linux-gnu (64-bit)
## 
## attached base packages:
## [1] compiler  stats     graphics  grDevices utils     datasets  methods  
## [8] base     
## 
## other attached packages:
## [1] rbenchmark_1.0.0 dplyr_0.2        magrittr_1.0.1   knitr_1.6       
## [5] vimcom_0.9-93    setwidth_1.0-3   colorout_1.0-3  
## 
## loaded via a namespace (and not attached):
##  [1] assertthat_0.1  codetools_0.2-8 digest_0.6.4    evaluate_0.5.5 
##  [5] formatR_0.10    highr_0.3       parallel_3.1.0  Rcpp_0.11.2    
##  [9] stringr_0.6.2   tools_3.1.0
\end{verbatim}
\end{kframe}
\end{knitrout}
\end{frame} 

\begin{frame}[fragile]
  \frametitle{DRUG}
  \begin{itemize}
    \item Future MeeetUp Topics:
      \begin{itemize}
        \item (Possible) iPython / R speaker for later in July
        \item We need others speakers!
      \end{itemize}
    \item MeetUp locations/times
  \end{itemize}
\end{frame} 
%}}}

\end{document}

