% ---------------------------------------------------------------------------- %
% file: dplyr_drug_talk.Rnw
% author: Peter DeWitt <peter.dewitt@ucdenver.edu>
%
% presentation on dplyr, and as a result also magrittr, for the Denver R User
% Group (DRUG) MeetUp on 1 July 2014.
%
% ---------------------------------------------------------------------------- %

\documentclass{beamer}\usepackage[]{graphicx}\usepackage[]{color}
%% maxwidth is the original width if it is less than linewidth
%% otherwise use linewidth (to make sure the graphics do not exceed the margin)
\makeatletter
\def\maxwidth{ %
  \ifdim\Gin@nat@width>\linewidth
    \linewidth
  \else
    \Gin@nat@width
  \fi
}
\makeatother

\definecolor{fgcolor}{rgb}{0.345, 0.345, 0.345}
\newcommand{\hlnum}[1]{\textcolor[rgb]{0.686,0.059,0.569}{#1}}%
\newcommand{\hlstr}[1]{\textcolor[rgb]{0.192,0.494,0.8}{#1}}%
\newcommand{\hlcom}[1]{\textcolor[rgb]{0.678,0.584,0.686}{\textit{#1}}}%
\newcommand{\hlopt}[1]{\textcolor[rgb]{0,0,0}{#1}}%
\newcommand{\hlstd}[1]{\textcolor[rgb]{0.345,0.345,0.345}{#1}}%
\newcommand{\hlkwa}[1]{\textcolor[rgb]{0.161,0.373,0.58}{\textbf{#1}}}%
\newcommand{\hlkwb}[1]{\textcolor[rgb]{0.69,0.353,0.396}{#1}}%
\newcommand{\hlkwc}[1]{\textcolor[rgb]{0.333,0.667,0.333}{#1}}%
\newcommand{\hlkwd}[1]{\textcolor[rgb]{0.737,0.353,0.396}{\textbf{#1}}}%

\usepackage{framed}
\makeatletter
\newenvironment{kframe}{%
 \def\at@end@of@kframe{}%
 \ifinner\ifhmode%
  \def\at@end@of@kframe{\end{minipage}}%
  \begin{minipage}{\columnwidth}%
 \fi\fi%
 \def\FrameCommand##1{\hskip\@totalleftmargin \hskip-\fboxsep
 \colorbox{shadecolor}{##1}\hskip-\fboxsep
     % There is no \\@totalrightmargin, so:
     \hskip-\linewidth \hskip-\@totalleftmargin \hskip\columnwidth}%
 \MakeFramed {\advance\hsize-\width
   \@totalleftmargin\z@ \linewidth\hsize
   \@setminipage}}%
 {\par\unskip\endMakeFramed%
 \at@end@of@kframe}
\makeatother

\definecolor{shadecolor}{rgb}{.97, .97, .97}
\definecolor{messagecolor}{rgb}{0, 0, 0}
\definecolor{warningcolor}{rgb}{1, 0, 1}
\definecolor{errorcolor}{rgb}{1, 0, 0}
\newenvironment{knitrout}{}{} % an empty environment to be redefined in TeX

\usepackage{alltt}

\usepackage{verbatim}

\author{Peter DeWitt\\peter.dewitt@ucdenver.edu}
\date{1 July 2014}
\title{Introduction to {\tt dplyr} and {\tt magrittr}}
\subtitle{Denver R Users Group\\www.meetup.com/DenverRUG}
\IfFileExists{upquote.sty}{\usepackage{upquote}}{}
\begin{document}



\begin{frame}[fragile]
  \maketitle
\end{frame} 

\begin{frame}[fragile]
  \frametitle{Goals:}

  \begin{itemize}
    \item Showcase {\tt dplyr}, compare the ease of use compared to base R.
    \item Introduce the data manipulation grammar and philosophy behind {\tt
      dplyr}
    \item Illustrate the usefulness of the forward-piping operator which is
      part of {\tt dplyr} and extended further in {\tt magrittr}.  
  \end{itemize}

  % \tableofcontents

\end{frame} 

\section{{\tt dplyr}}
\begin{frame}[fragile]
  \frametitle{{\tt dplyr}: a grammar of data manipulation}
  \begin{itemize}
    \item Authored by Hadley Wickham and Romain Francois
    \item Current CRAN version 0.2

    \item<2-> Paraphrasing from a post on the RStudio blog
      \url{http://blog.rstudio.org/2014/01/17/introducing-dplyr}

      \begin{itemize}
        \item {\tt dplyr} is the next iteration of {\tt plyr}
        \item focuses only on {\tt data.frame}s
        \item faster, thanks in part to Francois work in {\tt Rcpp}, some use of
          multiple processors.
        \item improved API. 
        \item interface with remote database (PostgreSQL, MySQL, SQLite, and
          Google bigquery) tables using the same verbs for
          interacting with {\tt data.frame}s.  (Extendible to other backends)
        \item Common operations:
          \begin{itemize}
            \item {\tt group\_by}, {\tt summarize}, {\tt mutate}, {\tt filter},
              {\tt select}, and {\tt arrange}.
          \end{itemize}
      \end{itemize}

  \end{itemize}
\end{frame} 

\subsection{Data Import}

\begin{frame}[fragile]
  \frametitle{Data Import}
  {\tt dplyr} does not have special tools for reading in data, but, if you need
  to {\tt rbind} sets together\ldots 



\begin{knitrout}\footnotesize
\definecolor{shadecolor}{rgb}{0.969, 0.969, 0.969}\color{fgcolor}\begin{kframe}
\begin{alltt}
\hlcom{# FAAs wildlife strikes on aircraft since 1990.  The data}
\hlcom{# can be downloaded, in a Microsoft Access DB,  from}
\hlcom{# http://www.faa.gov/airports/airport_safety/wildlife/database/}
\hlcom{# Tables in the DB were exported to csv files.  }
\hlcom{# A data dictionary, in an Excel file, was also}
\hlcom{# included in the download from faa.gov}

\hlcom{# column classes are set (in R code not shown) to ensure}
\hlcom{# that each column of the imported data is of the same class}
\hlstd{wls.90.99} \hlkwb{<-}
  \hlkwd{read.csv}\hlstd{(}\hlstr{"../data/STRIKE_REPORTS (1990-1999).csv"}\hlstd{,}
           \hlkwc{colClasses} \hlstd{= clclss)}
\hlstd{wls.00.09} \hlkwb{<-}
  \hlkwd{read.csv}\hlstd{(}\hlstr{"../data/STRIKE_REPORTS (2000-2009).csv"}\hlstd{,}
           \hlkwc{colClasses} \hlstd{= clclss)}
\hlstd{wls.10.14} \hlkwb{<-}
  \hlkwd{read.csv}\hlstd{(}\hlstr{"../data/STRIKE_REPORTS (2010-Current).csv"}\hlstd{,}
           \hlkwc{colClasses} \hlstd{= clclss)}
\end{alltt}
\end{kframe}
\end{knitrout}
\end{frame} 


\begin{frame}[fragile]
  \frametitle{Data Import}
\begin{knitrout}\footnotesize
\definecolor{shadecolor}{rgb}{0.969, 0.969, 0.969}\color{fgcolor}\begin{kframe}
\begin{alltt}
\hlcom{# Base does not require the columns to be of the same class,}
\hlcom{# only the same name}
\hlcom{# dplyr requires that the columnns are of the same class.}
\hlkwd{dim}\hlstd{(wls.90.99)}
\end{alltt}
\begin{verbatim}
## [1] 30150    94
\end{verbatim}
\begin{alltt}
\hlkwd{nrow}\hlstd{(wls.90.99)} \hlopt{+} \hlkwd{nrow}\hlstd{(wls.00.09)} \hlopt{+} \hlkwd{nrow}\hlstd{(wls.10.14)}
\end{alltt}
\begin{verbatim}
## [1] 142911
\end{verbatim}
\begin{alltt}
\hlstd{bnchmrk} \hlkwb{<-}
  \hlkwd{benchmark}\hlstd{(}\hlkwc{base} \hlstd{=} \hlkwd{rbind}\hlstd{(wls.90.99, wls.00.09, wls.10.14),}
            \hlkwc{dplyr} \hlstd{=} \hlkwd{rbind_list}\hlstd{(wls.90.99, wls.00.09, wls.10.14),}
            \hlkwc{replications} \hlstd{=} \hlnum{100}\hlstd{)}
\hlstd{bnchmrk[,} \hlkwd{c}\hlstd{(}\hlstr{"test"}\hlstd{,} \hlstr{"replications"}\hlstd{,} \hlstr{"elapsed"}\hlstd{,} \hlstr{"relative"}\hlstd{)]}
\end{alltt}
\begin{verbatim}
##    test replications elapsed relative
## 1  base          100   87.99    3.878
## 2 dplyr          100   22.69    1.000
\end{verbatim}
\end{kframe}
\end{knitrout}
\end{frame} 

\begin{frame}[fragile]
  \frametitle{Data Import}
\begin{knitrout}\footnotesize
\definecolor{shadecolor}{rgb}{0.969, 0.969, 0.969}\color{fgcolor}\begin{kframe}
\begin{alltt}
\hlstd{wls_df} \hlkwb{<-} \hlkwd{rbind}\hlstd{(wls.90.99, wls.00.09, wls.10.14)}
\hlkwd{class}\hlstd{(wls_df)}
\end{alltt}
\begin{verbatim}
## [1] "data.frame"
\end{verbatim}
\begin{alltt}
\hlstd{wls} \hlkwb{<-} \hlkwd{rbind_list}\hlstd{(wls.90.99, wls.00.09, wls.10.14)}
\hlkwd{class}\hlstd{(wls)}
\end{alltt}
\begin{verbatim}
## [1] "data.frame"
\end{verbatim}
\begin{alltt}
\hlcom{# A data frame tbl wraps a local data frame. The main}
\hlcom{# advantage to using a ‘tbl_df’ over a regular data frame is}
\hlcom{# the printing: tbl objects only print a few rows and all}
\hlcom{# the columns that fit on one screen, providing describing}
\hlcom{# the rest of it as text. [source: R help doc]}
\hlstd{wls_tbl_df} \hlkwb{<-} \hlkwd{tbl_df}\hlstd{(wls)}
\hlkwd{class}\hlstd{(wls_tbl_df)}
\end{alltt}
\begin{verbatim}
## [1] "tbl_df"     "tbl"        "data.frame"
\end{verbatim}
\end{kframe}
\end{knitrout}
\end{frame} 

\begin{frame}[fragile]
  \frametitle{Data Printing}
\begin{knitrout}\footnotesize
\definecolor{shadecolor}{rgb}{0.969, 0.969, 0.969}\color{fgcolor}\begin{kframe}
\begin{alltt}
\hlcom{# print(wls_df)  # takes a long time, not helpful}
\hlcom{# head(wls_df)   # two many columns to be useful}
\hlkwd{print}\hlstd{(wls_tbl_df,} \hlkwc{n} \hlstd{=} \hlnum{2}\hlstd{)}
\end{alltt}
\begin{verbatim}
## Source: local data frame [142,911 x 94]
## 
##    INDEX_NR OPID          OPERATOR     ATYPE AMA AMO EMA EMO
## 1    100000  AAL AMERICAN AIRLINES     B-727 148  10  34  10
## 2    100001  UAL   UNITED AIRLINES B-737-300 148  24  10  01
## ..      ...  ...               ...       ... ... ... ... ...
## Variables not shown: AC_CLASS (chr), AC_MASS (int), NUM_ENGS
##   (chr), TYPE_ENG (chr), ENG_1_POS (chr), ENG_2_POS (int),
##   ENG_3_POS (chr), ENG_4_POS (int), REG (chr), FLT (chr),
##   REMAINS_COLLECTED (lgl), REMAINS_SENT (lgl), INCIDENT_DATE
##   (chr), INCIDENT_MONTH (int), INCIDENT_YEAR (int),
##   TIME_OF_DAY (chr), TIME (int), AIRPORT_ID (chr), AIRPORT
##   (chr), STATE (chr), FAAREGION (chr), ENROUTE (chr), RUNWAY
##   (chr), LOCATION (chr), HEIGHT (int), SPEED (int), DISTANCE
##   (dbl), PHASE_OF_FLT (chr), DAMAGE (chr), STR_RAD (lgl),
##   DAM_RAD (lgl), STR_WINDSHLD (lgl), DAM_WINDSHLD (lgl),
##   STR_NOSE (lgl), DAM_NOSE (lgl), STR_ENG1 (lgl), DAM_ENG1
##   (lgl), STR_ENG2 (lgl), DAM_ENG2 (lgl), STR_ENG3 (lgl),
##   DAM_ENG3 (lgl), STR_ENG4 (lgl), DAM_ENG4 (lgl), INGESTED
##   (lgl), STR_PROP (lgl), DAM_PROP (lgl), STR_WING_ROT (lgl),
##   DAM_WING_ROT (lgl), STR_FUSE (lgl), DAM_FUSE (lgl), STR_LG
##   (lgl), DAM_LG (lgl), STR_TAIL (lgl), DAM_TAIL (lgl),
##   STR_LGHTS (lgl), DAM_LGHTS (lgl), STR_OTHER (lgl),
##   DAM_OTHER (lgl), OTHER_SPECIFY (chr), EFFECT (chr),
##   EFFECT_OTHER (chr), SKY (chr), PRECIP (chr), SPECIES_ID
##   (chr), SPECIES (chr), BIRDS_SEEN (chr), BIRDS_STRUCK (chr),
##   SIZE (chr), WARNED (chr), COMMENTS (chr), REMARKS (chr),
##   AOS (int), COST_REPAIRS (int), COST_OTHER (int),
##   COST_REPAIRS_INFL_ADJ (int), COST_OTHER_INFL_ADJ (int),
##   REPORTED_NAME (chr), REPORTED_TITLE (chr), REPORTED_DATE
##   (chr), SOURCE (chr), PERSON (chr), NR_INJURIES (int),
##   NR_FATALITIES (int), LUPDATE (chr), TRANSFER (lgl),
##   INDICATED_DAMAGE (lgl)
\end{verbatim}
\end{kframe}
\end{knitrout}
\end{frame} 


\section{magrittr}%{{{
\begin{frame}[fragile]
  \frametitle{{\tt magrittr}: a forward-pipe operator for {\tt R}}
  \framesubtitle{ceci n'est pas un pipe (this is not a pipe)}

  \begin{itemize}
    \item {\tt dplyr} funcationality is made more powerful via the \verb|%>%|,
      or equivalently, \verb|\%.%$|, operator.

      %     \item<2-> Additional functionally provided by the {\tt magrittr} package
      %       authored by Stefan Bache and Hadley Wickham.
      % 
      %     \item<3-> These operators are similar to 
      %       \begin{itemize} 
      %         \item F\#'s $|>$, or
      %         \item Linux's $|$.
      %       \end{itemize}
      % 
      %     \item<4-> Use of these operators will drastically change your {\tt R} syntax.
      % 
      %     \item<5-> Helpful to writting complex, nested, operations.
      %     \item<5->``Read from left to right instead of inside out.''

  \end{itemize}
\end{frame} 

\begin{frame}[fragile]
  \frametitle{{\tt magrittr}: a foward-pipe operator for {\tt R}}
  \framesubtitle{Examples} 

\begin{knitrout}\footnotesize
\definecolor{shadecolor}{rgb}{0.969, 0.969, 0.969}\color{fgcolor}\begin{kframe}
\begin{alltt}
\hlkwd{data}\hlstd{(diamonds,} \hlkwc{package} \hlstd{=} \hlstr{"ggplot2"}\hlstd{)}

\hlcom{# find the mean price of the diamonds}
\hlcom{# Standard R syntax}
\hlkwd{mean}\hlstd{(diamonds}\hlopt{$}\hlstd{price)}
\end{alltt}
\begin{verbatim}
## [1] 3933
\end{verbatim}
\begin{alltt}
\hlcom{# with the pipe}
\hlstd{diamonds} \hlopt
\hlkwd{extract}\hlstd{(}\hlstr{"price"}\hlstd{)} \hlopt
\hlkwd{unlist}\hlstd{()} \hlopt
\hlkwd{mean}\hlstd{()}
\end{alltt}
\begin{verbatim}
## [1] 3933
\end{verbatim}
\end{kframe}
\end{knitrout}

What's the point?
\end{frame} 
%}}}

\begin{frame}[fragile]
  \frametitle{Reproducibility}
  The data, code, sides, etc.\ all at \url{github.com/dewittpe/dplyr-demo}

\begin{knitrout}\footnotesize
\definecolor{shadecolor}{rgb}{0.969, 0.969, 0.969}\color{fgcolor}\begin{kframe}
\begin{alltt}
\hlkwd{print}\hlstd{(}\hlkwd{sessionInfo}\hlstd{(),} \hlkwc{locale} \hlstd{=} \hlnum{FALSE}\hlstd{)}
\end{alltt}
\begin{verbatim}
## R version 3.1.0 (2014-04-10)
## Platform: x86_64-pc-linux-gnu (64-bit)
## 
## attached base packages:
## [1] stats     graphics  grDevices utils     datasets 
## [6] methods   base     
## 
## other attached packages:
## [1] rbenchmark_1.0.0 dplyr_0.2        magrittr_1.0.1  
## [4] knitr_1.6        vimcom_0.9-93    setwidth_1.0-3  
## [7] colorout_1.0-3  
## 
## loaded via a namespace (and not attached):
##  [1] assertthat_0.1  codetools_0.2-8 digest_0.6.4   
##  [4] evaluate_0.5.5  formatR_0.10    highr_0.3      
##  [7] parallel_3.1.0  Rcpp_0.11.1     stringr_0.6.2  
## [10] tools_3.1.0
\end{verbatim}
\end{kframe}
\end{knitrout}
\end{frame} 
\end{document}

